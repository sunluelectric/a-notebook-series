\chapter*{Preface}

Artificial intelligence (AI) was originally a field within control systems, primarily used for system identification. Its artificial neuron network structure provided a promising technology for building highly nonlinear and self-tuning functions.

With the advancement of computing hardware, particularly GPU and TPU, artificial neuron networks have become exceptionally comprehensive. Today, it is common to find networks with dozens of layers, each containing hundreds of neurons, resulting in deep learning networks. Alongside the emergence of deep learning networks, convolutional and recurrent neural networks have proven to be efficient in identifying hidden items in images and trends. The invention of transformers, another deep learning structure that is purely attention-based, has taken natural language processing to the next level.

AI has grown so significantly that it is now considered a separate subject from control systems. To put it in perspective, a classic textbook on control systems may have 10,000+ citations, while a famous conference paper on modern AI can easily have 50,000+ citations.

This notebook does not focus on introducing AI mechanisms, as they are already covered in control system related notebooks. The purpose of this notebook is mainly to keep up-to-date with the latest technology in AI.

\vspace{.2in}

\noindent \textbf{(The above paragraphs are polished by ChatGPT v3.)}
