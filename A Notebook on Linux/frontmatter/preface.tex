\chapter*{Preface}

I have used Microsoft Windows for decades in both daily life and research, and it has served me well. Windows offers an intuitive user interface and supports a wide range of software, including Adobe, MATLAB, WinEdt, Steam, EA Games, and the Edge web browser. One could live without ever touching Linux and still get along just fine.

Some may argue that Linux (and also macOS) is primarily appreciated by ``professional users'' such as developers and data centers servers. However, this view is not entirely accurate. Today, many tools used by data scientists and developers, such as database servers and integrated development environments, are readily available on Windows, with some of the best ones even developed by Microsoft. Developers can work productively in countless ways without ever needing to interact with Linux directly. In fact, many industrial computers ship with Windows pre-installed and run reliably. If you interview front-line factory operators, how many of them would use Linux commands on a daily basis? As long as the graphical interfaces work properly, few users are concerned about whether the backend is running Windows or Linux.

That said, learning Linux remains important, especially for researchers and professionals with an engineering or technical background.

Compared to Windows, Linux is open-source, more flexible, and highly customizable. It has a vibrant global community that actively supports its development. For certain tasks, it outperforms Windows. In some cases, software tools are easier to install and use on Linux, thanks to fewer restrictions and a more transparent system architecture.

Even if we do not interact with Linux directly on a daily basis, we frequently rely on software and tools that run on Linux environments. For example, Android is built on the Linux kernel. Containerization technologies such as Docker, Podman and Kubernetes run natively on Linux. When using them on Windows machines, they typically create a hidden Linux virtual machine in the backend.

Beyond these reasons, the main motivation for this notebook is to show that learning Linux is not just about mastering another operating system. It provides a strong foundation for understanding computer operating systems in general, including Windows, macOS, UNIX and more. Gaining a deeper understanding of how computers and digital systems work can genuinely improve productivity. After all, we interact with these systems every day. The next time your computer slows down, at least you will check your CPU and memory usage before jumping into buying a new one.

A list of key references of this notebook is given below. These materials are frequently cited in the notebook and it will be tedious to mark each and every of their appearances. For convenience, they are not given in the Reference, but listed here instead.

\begin{itemize}
	\item The Linux Bible (10th edition)
	\item Docker \& Kubernetes: The Practical Guide (2024 edition), Udemy
\end{itemize}
 
 The majority of the content in this notebook was tested on a machine with Red Hat Enterprise Linux 9, with some exceptions from Ubuntu. 