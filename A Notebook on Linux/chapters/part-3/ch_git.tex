\chapter{Git} \label{ch:git}

Git is a distributed version control system for tracking changes during the software development, and it can be used on multiple platforms including Windows, Unix/Linux and MacOS. This chapter introduces the basic use of Git on a local Linux machine and with a remote repository host server such as GitHub. 

Git-based version control is widely used in association with \myabb{Continuous Integration and Continuous Deployment}{CI/CD}. CI/CD is an important concept in nowadays software development and deployment. GitHub Actions is the tool that GitHub manages its CI/CD workflow, and its basics are also introduced.

\section{Introduction}

\mync{Git}, created by Linus Trovalds in 2005, is a distributed version control system for tracking changes in source code and files. It is helpful with maintaining data integrity during the collaborative development of software in distributed non-linear workflows. Git is free and open-source under GNU general public license.

Git introduces \mync{Git repository}, which refers to the ``ecosystem'' Git uses to manage a project. To be more specific, it refers to the \verb|.git| file associated with each Git-managed project, using which Git traces the development of the project. 

With Git, all computers participating in the software development store a copy of the full-fledged Git repository locally with complete history, and it can synchronize with a centralized remote server. It introduces \mync{branch}, to be more precise ``master'' and ``slave'' branches, to manage the concurrent development of different features of the project, where the master branch is the stable and shared repository among everyone, and the feature branches are copies of the master branch where individual features can be developed. For a feature branch, once its developed feature is approved, it can be merged back to the master branch. A demonstration is given in Fig. \ref{ch:sma:fig:gitflow}.
\begin{figure}[htbp]
	\centering
	\includegraphics[width=350pt]{chapters/part-3/figures/gitflow.png}
	\caption{Git for software development management.} \label{ch:sma:fig:gitflow}
\end{figure}

CLI is often used to manage a Git repository. Some commonly used commands are shown in Fig. \ref{ch:sma:fig:gitflow} with more to be introduced later. Notice that a graphical user interface is also available. However, in the scope of this notebook, command line is mostly used.

\section{Git Installation and Configuration}

Git can be installed and configured as follows.

\subsection{Installation}

Git and its relevant documents can be obtained from its official website \textit{https://git-scm.com/}. In many Linux distributions, Git is pre-installed. If Git is missing, follow the instructions on the official website to install it. Notice that the installation procedure may differ with different OS.

To install Git on RHEL, simply use
\begin{lstlisting}
$ sudo dnf install git-all
\end{lstlisting}

\subsection{Configuration}

Upon successful installation, it is recommended to use \verb|git config| for the necessary basic configurations such as the username and user email. Without these configurations, Git cannot run normally.

Notice that there are two types of configurations, namely the global configurations which apply to the machine and the user, and the repository configurations which apply to a particular Git repository. By default, the global level configurations are stored under \verb|~/.gitconfig| and the repository configurations \verb|./.git/config| of the repository, respectively. It is recommended to use Git CLI to setup the configurations, rather than modifying the files directly.

To add user name and email to the global configuration, use
\begin{lstlisting}
$ git config --global user.name '<user name>'
$ git config --global user.email '<user email>'
\end{lstlisting}
To retrieve the global configuration, use
\begin{lstlisting}
$ git config --global -l
\end{lstlisting}
To revoke a global configuration, use
\begin{lstlisting}
$ git config --global --unset <configuration>
\end{lstlisting}
For example,
\begin{lstlisting}
$ git config --global --unset user.name
\end{lstlisting}
removes the user name.

More details about \verb|git config| can be found at \cite{git2025reference}.

\section{Local Repository Management}

Git CLI can manage both local and remote repositories, and can synchronize them bidirectionally. This section studies local repository management.

\subsection{Initialization of a Repository}

Navigate to the project directory. Use the following command to create a new Git repository for the project.
\begin{lstlisting}
$ git init
\end{lstlisting}
From this point forward, Git monitors everything that happens inside this directory and its sub-directories and tries to track any change to the files, unless otherwise configured specifically. Many Git commands such as \verb|git status| become available. More details are introduced in the remaining of the section.

\subsection{Version Tracking}

For simplicity, assume that there is only one branch in the repository, namely the master branch. Notice that when there are multiple branches, the version-tracking works the same for each and every branch in a separate and independent manner. The mechanism behind version tracking is briefly introduced as follows.

The project directory is split into two parts, outside \verb|./.git/| the workspace, and inside \verb|./.git/| the Git repository. The workspace has the up-to-date project contents and it is directly managed by the user, while Git repository is managed by Git. The user should not manage the repository directly unless using the Git interface.

Inside the Git repository are metadata of the workspace files such as which files have been changed since the last deployment, etc. It also stores a full back up of every historical versions of the project (with powerful compressing mechanisms, of course). It is worth mentioning that instead of recording the changes of a file from version to version, Git records the snapshot of the file in every version, unless it is left untouched between two consecutive versions.

Figure \ref{ch:sma:fig:gitbasics} gives a demonstration of how Git manages the project directory. Git categorizes files in the workspace into the following states: untracked, modified (unstaged), staged, and unmodified. This is shown by Fig. \ref{ch:sma:fig:gitbasics}. A brief explanation of each state is given in Table \ref{ch:sma:tab:gitbasics}. More details are given later.

\begin{figure}[!htb]
	\centering
	\includegraphics[width=350pt]{chapters/part-3/figures/gitbasics.png}
	\caption{The project directory managed by Git.} \label{ch:sma:fig:gitbasics}
\end{figure}

\begin{table}[!htb]
	\centering \caption{Different file status in a Git managed project.}\label{ch:sma:tab:gitbasics}
	\begin{tabularx}{\textwidth}{lX}
		\hline
		Status & Description \\ \hline
		Untracked & Newly added or renamed items in the project directory.  \\ 
		Modified (Unstaged) & Modified items from the last version that has not been registered in the staging area.  \\ 
		Modified (Staged) & Modified items from the last version that has been registered in the staging area. Notice that an untracked item can be staged directly, skipping ``modified (unstaged)'' step. Use \verb|git add| to stage items. \\ 
		Unmodified & Unmodified items from the last commit. \\ \hline
	\end{tabularx}
\end{table}

Use \verb|git status| to check the file states in the project. An example is given below.
\begin{lstlisting}
$ git status
On branch master

Changes to be committed:
  (use "git restore --staged <file>..." to unstage)
	modified:   chapters/ch-software-management-advanced/ch.tex

Changes not staged for commit:
  (use "git add <file>..." to update what will be committed)
  (use "git restore <file>..." to discard changes in working directory)
	modified:   appendix/ap.tex
	modified:   main.pdf

Untracked files:
  (use "git add <file>..." to include in what will be committed)
	chapters/ch-software-management-advanced/figures/
\end{lstlisting}
where \verb|ch.tex| is a modified (staged) item; \verb|ap.tex| and \verb|main.pdf| modified (unstaged) items, and \verb|figures/| an untracked item. Notice that unmodified files are not shown.

It takes a two-step procedure to back up the project in the Git repository. In the first step, the user flags the changed items (either newly added, renamed, or modified) to be backed up in the next version. In the second step, the user actually backs up the items. The first and second steps are called ``stage'' and ``commit'' respectively. Notice that it is possible to run a single line of command to execute both steps, but logically it still takes two steps.

Git tracks the name and content of the items that the user has staged in the ``staging area'' as shown in Fig. \ref{ch:sma:fig:gitbasics}. Think of staging items as taking a snapshot of the items. However, the snapshot at this stage is temporary and has not been backed up in the repository yet. The actual backup happens later when the staged items are committed. To stage an item, use
\begin{lstlisting}
$ git add <item name>
\end{lstlisting}
which registers the item in the staging area, thus also changes its status from untracked or modified (unstaged) to modified (staged). If an item is modified after it has been staged, Git will distinguish the ``staged portion'' and ``unstaged portion'' of that item. If using \verb|git status| to check its status, the item will be listed as both staged and unstaged. Unstaged items, either untracked or modified, will remain its status after the commit. Sometimes for convenience, \verb|git add -A| can be used to add all untracked or modified items to the staging area.

Use \verb|git commit| to commit the staged items and back them up in the repository as follows.
\begin{lstlisting}
$ git commit [<item name>]
\end{lstlisting}
The above command commits the project and creates a version in the Git repository. It is possible to specify items, in which case Git only commits the specified items and leave the rest items as they are. A commit ID is automatically assigned to the commit. Notice that the user will be asked to provide a ``comment message'' with the commit, which should be used to briefly explain what has been changed in this commit.

A flag \verb|-a| with \verb|git commit| stages all changes made to the project, then implements the commit command. A flag \verb|-m| simplifies the message recording process and allows the user to key in the message directly after the command. An example is given below.
\begin{lstlisting}
$ git status
On branch master

Changes not staged for commit:
  (use "git add <file>..." to update what will be committed)
  (use "git restore <file>..." to discard changes in working directory)
        modified:   A Notebook on Linux/chapters/ch-software-management-advanced/ch.tex
        modified:   A Notebook on Linux/main.pdf

no changes added to commit (use "git add" and/or "git commit -a")

$ git commit -am "add introduction to git command"
[master e2e977e] add introduction to git command
 2 files changed, 5 insertions(+), 4 deletions(-)

$ git status
On branch master

nothing to commit, working tree clean
\end{lstlisting}

To check the commit logs, i.e., all historical commits including their associated timestamps, authors, commit IDs and comment messages, use \verb|git log| as shown in the example below. Notice that the commit logs can be very long. Only a few commits are given for illustration purpose.
\begin{lstlisting}
$ git log
commit e3475e673d8c2a087de6b4423188c51e80af3e5d (HEAD -> master)
Author: sunlu <sunlu.electric@gmail.com>
Date:   Wed Aug 31 15:16:52 2022 +0800

    git

commit 3ab6d1473a7e48d4d890509ddb5a87274c023e6c
Author: sunlu <sunlu.electric@gmail.com>
Date:   Tue Aug 30 15:50:06 2022 +0800

    k8s

commit f6a1e3d779305e966602ecd7589c05f56dd2ad0f
Author: sunlu <sunlu.electric@gmail.com>
Date:   Mon Aug 29 16:31:18 2022 +0800

    more on docker and kubernetes

commit e2de5b28db7982f57d0ad51361d5161b884f2efe
Author: sunlu <sunlu.electric@gmail.com>
Date:   Sun Aug 28 21:38:48 2022 +0800

    add docker sections
\end{lstlisting}
where notice that \verb|HEAD| is a reference that points to the latest commit in a branch. Filters can be added as optional arguments for the \verb|git log| command. More details are given at \cite{git2025reference}.

And finally, to restore to a previous commit version, use \verb|git reset| or \verb|git revert|. Notice that \verb|git reset| and \verb|git revert| can both be used to restore the workspace to a previous stage, but they differ significantly. Their differences are introduced below.

Command \verb|git reset| erases the past commits in the local branch. This should not pose any problem if the erased commits have not spread everywhere else. Therefore, \verb|git reset| is helpful for local repositories without upstreams, or for modifications that have not been pushed to the upstreams. 

The command \verb|git reset| is often used as follows.
\begin{lstlisting}
	$ git reset <option> <commit ID>
\end{lstlisting}
where \verb|<option>| is often \verb|--hard|, \verb|--mixed| or \verb|--soft|, and \verb|<commit ID>| can be the ID of any commit in the git log, or for shortcut \verb|HEAD| the current position of the HEAD (commit pointer), usually the latest commit.

The options \verb|--hard|, \verb|--mixed| and \verb|--soft| work differently. All these options move HEAD to the specified commit, and remove all commits afterwards from the repository. The \verb|--hard| option reverts the workspace back to when the specified commit happened (meaning that there would be no way to undo the reset command). Both \verb|--mixed| and \verb|--soft| do not change the workspace. The \verb|--mixed| option leaves all the changes from the specified commit to today as unstaged, while \verb|--soft| leaves them as staged. If no \verb|--hard|, \verb|--mixed| or \verb|--soft| is given, \verb|--mixed| will be used as the default option.

Notice that \verb|git reset| may pose an issue if the commits to be erased have already spread to the upstreams. This is because the erased commits in the local repository by \verb|git reset| will be reverted back the next time the local repository synchronizes with its upstream, where the commits in the upstream will be considered more up to the date, hence pulled to the local repository.

In such cases, consider using \verb|git revert| instead. Syntax wise they work similarly. Instead of erase the history, \verb|git revert| creates a new commit whose content is copied and pasted from a past commit.

\subsection{Branch Management}

\mync{Git branch} is a core concept and feature of Git, and it plays an important role in collaborative development of a project. 

There are two types of branches, namely the local branch and the remote branch. Remote branch is often associated with local branches, and play as the upstream mirror of the later for sharing and synchronizing the project development across machines. This section focuses on local branches. Remote branches will be introduced in Section \ref{ch:sma:sec:rrm}.

To list down all the local branches, use
\begin{lstlisting}
$ git branch
\end{lstlisting}
and the current working branch will be highlighted. The current working branch is also referred as the ``head branch'' or the ``active branch''.

To create a new branch from the current branch, and to delete a branch, use
\begin{lstlisting}
$ git branch <new branch name> [<commit ID>]
$ git branch -d <branch name>
\end{lstlisting}
respectively. The optional \verb|<commit ID>| when creating a new branch allows the user to create a branch on top of a specified historical commit instead of the latest commit.

To rename a branch, use
\begin{lstlisting}
$ git branch -m [<old branch name>] <new branch name>
\end{lstlisting}
where if no \verb|<old branch name>| is specified, the current working branch will be renamed.

To switch to a different working branch, use \verb|git checkout| or \verb|git switch| as follows.
\begin{lstlisting}
$ git checkout <branch name>
$ git switch <branch name>
\end{lstlisting}
Notice that when there are uncommitted changes in the current branch, Git may forbid the user to switch to another branch (the rules are too complicated to be explained here). Therefore, it is recommended to commit the changes before switching to a different branch. When switching to another branch, the workspace will change accordingly to the target branch.

To merge a branch back to the current branch, use
\begin{lstlisting}
$ git merge <branch name>
\end{lstlisting}
and fill in the comments accordingly. A use case of the command is to merge the features developed in a feature branch to the master branch, after a pull request is raised by the feature branch. To do that, checkout to the master branch, and use
\begin{lstlisting}
$ git merge <feature branch name>
\end{lstlisting}
By default, Git automatically deletes the merged branch. This behavior can be changed in the configuration. More about pull request are introduced in Section \ref{sec:pull_request}.

An alternative way of integrating two branches is \verb|git rebase|, which ``rewires'' the two branches into a linear single branch. Use \verb|git rebase| as follows.
\begin{lstlisting}
$ git rebase <branch name>
\end{lstlisting}
A demonstrative Fig. \ref{ch:sma:fig:gitrebase} explains the differences between \verb|git merge| and \verb|git rebase|. It is clear from Fig. \ref{ch:sma:fig:gitrebase} that by using \verb|git rebase|, all feature commits are integrated into the master repositories to form a single repository tracking line, which differs largely from \verb|git merge|.
\begin{figure}[!htb]
	\centering
	\includegraphics[width=350pt]{chapters/part-3/figures/gitrebase.png}
	\caption{Two approaches of integrating branches, \texttt{git merge} VS \texttt{git rebase}. In the \texttt{git rebase} example, two rebase commands are executed consecutively, the first on the feature branch and second the master branch.} \label{ch:sma:fig:gitrebase}
\end{figure}

An intuitive way to understand \verb|git rebase| is given as follows. Imagine two branches $A$ and $B$ that have diverged from the same commit $O$. On branch $A$, execute \verb|git rebase B|. Here is a step-by-step breakdown of what happens:
\begin{enumerate}
	\item Identify common ancestor of the current branch $A$ and the rebase branch $B$, which is commit $O$ in this example.
	\item Back up commits happened on branch $A$ since commit $O$. Assume they are commits $A_1$, $A_2$, ..., $A_n$.
	\item Reset branch $A$ to Branch $B$ and $A$ now starts from the tip of $B$.
	\item Reapply commits $A_1$, $A_2$, ..., $A_n$ on the reset $A$ in a sequential manner. In the case where there are conflicts when reapplying $A_1$, $A_2$, ..., $A_n$, the user needs to jump in and resolve them manually.
\end{enumerate}
After the rebase operation, the history of branch $A$ appears as if it was developed sequentially from the end of branch $B$, effectively integrating the changes of $B$ into $A$ in a linear history.

\section{Remote Repository Management} \label{ch:sma:sec:rrm}

Git hosting servers provide platforms for the developers to share their Git repositories and collaboratively develop projects. There are many cloud-based public Git hosting service providers such as GitHub, GitLab and Gitee. In this section, GitHub is studied. Notice that under the scope of this section, only basic functions are introduced and they should apply similarly to other Git hoisting service providers.   

An HTTPS URL is associated with each remote repository on GitHub, for example \textit{https://github.com/torvalds/linux.git} for Linux kernel. This URL can be used to download the remote repository to a local machine, or to link (synchronize) a local repository with that remote repository.

\subsection{Clone and Pull: from Remote to Local}

Consider the case where there is already a remote repository, either under the user's GitHub account or from the community. The user can clone the repository, i.e., make a local copy of the repository, as follows.

Use
\begin{lstlisting}
$ git clone <repository URL> [<local directory>]
\end{lstlisting}
where \verb|repository URL| can be the HTTPS URL of the repository. 

Git is able to trace the URL from which a local repository is cloned. To maintain the local repository up-to-date with the remote repository, regularly use
\begin{lstlisting}
$ git remote update
$ git pull
\end{lstlisting}
to pull the latest update.

\begin{shortbox}
\Boxhead{Should I use \texttt{git pull} or \texttt{git fetch}?}

One may argue whether it is a good idea to use \verb|git pull|. Although convenient, \verb|git pull| is an integration of two commands, \verb|git fetch| and \verb|git merge| (or \verb|git rebase|) executed sequentially. It may be ``safer'' to manually run the two commands separately.

With that being said, if the user simply wants to keep a well-maintained community project up to date, \verb|git pull| should work just alright.
\end{shortbox}

However, this does not guarantee that the user can perform a \verb|git push| command to push changes made in the local repository back to the remote source. For that, additional authentication is often required. More about \verb|git push| and remote repository authentication methods are introduced in later sections.

\subsection{Push: from Local to Remote}

Consider a scenario where you have a well-developed local Git repository, and you have just created an empty remote repository on GitHub. The goal is to link the remote repository with the local repository so that changes can be synchronized.

To establish a connection between the local and remote repositories, navigate to the local repository and run
\begin{lstlisting}
$ git remote add <remote-name> <remote-repository-URL>
\end{lstlisting}
This command registers the remote repository's URL in the local Git configuration. A commonly used remote name is ``origin''.

Note that if the local repository was originally created by cloning an existing remote repository, then ``origin'' is likely already set to the source of the clone. Use \verb|git remote -v| to check the registered remote source.

The next step is to associate the current working branch with a branch in the remote repository. This ensures that git pull and git push operate on the correct branch by default. This can be done as follows.
\begin{lstlisting}
$ git branch --set-upstream-to=<remote-name>/<branch-name>
\end{lstlisting}
Alternatively, the upstream branch can be set when pushing it for the first time using
\begin{lstlisting}
$ git push --set-upstream <remote-name> <branch-name>
\end{lstlisting}

Once the upstream is configured, the user can use 
\begin{lstlisting}
$ git push
\end{lstlisting}
to push the changes in the local remository to the remote upstream. If there is no remote branch with the same branch name as the local repository, a remote branch with that name will be created when the local branch is pushed.

\begin{shortbox}
\Boxhead{The Names of Corresponding Local and Remote Branches}

A local branch and its corresponding remote branch usually have the same name by default, but this is not necessarily always the case. To set an upstream with a different name, use
\begin{lstlisting}
$ git branch --set-upstream-to=\
<remote-name>/<remote-branch-name> <local-branch-name>
\end{lstlisting}
or
\begin{lstlisting}
$ git push <remote-name> <local-branch-name>:<remote-branch-name>
\end{lstlisting}

\end{shortbox}

When performing \texttt{git push} for the first time, authentication is required. This allows GitHub or another hosting service to verify the identity of the user before accepting changes. Common authentication methods include
\begin{itemize}
    \item Username and password authentication. Notice that this has been deprecated by GitHub in favor of token authentication.
    \item Personal access tokens.
    \item SSH key authentication.
\end{itemize}

\section{Pull Request} \label{sec:pull_request}

In a collaborative project, the master branch is managed very carefully because everything on that branch will affect the project in the production environment. There is often a group of senior developers maintaining the master branch. Any code to be added to the master branch must be reviewed by one or a few of them.

When a feature branch wants the features developed on that branch to be merged to the master branch, it does not push the changes to the master branch directly. Instead, the owner of the feature branch needs to raise a ``pull request'' from the GitHub dashboard. As its name suggests, it requests the master branch to pull the updates from the feature branch. The master branch maintenance team will then view and check the feature branch, making sure that everything is correct before they approve the merging.

If the pull request is approved, GitHub will perform the merge in the cloud. Neither the owner of the feature branch nor the maintenance team needs to run the merge manually.

\section{Collaborative Development in Practice}

This section introduces the common procedures an individual contributor follows to make improvements to a community project. We assume that an individual contributor has found a community project hosted on GitHub that interests him and believes he can update the code to improve it.

The following is a general flow where an individual contributor helps with improving an open-source project.
\begin{enumerate}
	\item The contributor forks the project to his own GitHub account.
	\item The contributor clones a copy of the project from his own GitHub account to his own machine.
	\item The contributor creates a new branch about the feature he wants to add or improve.
	\item The contributor develops the feature on the branch.
	\item The contributor commits the development and pushes the commit to the forked repository.
	\item The contributor submits a pull request to the maintenance team of the community project.
	\item The maintenance team receives the pull request and reviews the changes made to the code in the developer's forked repository.
	\item The maintenance team either sends feedback to the contributor for further clarification, or pulls the changes from the developer's forked repository and merge the new feature branch with the existing branches.
\end{enumerate}

As a first step, the contributor forks the project into his own GitHub account. This creates a copy of the community project under the contributor’s account, giving him full control over the forked repository. He is free to make or test any modifications without affecting the original project.

Although a fork is treated as a separate repository in many ways, GitHub tracks the relationship between the fork and the original project. This allows the contributor to later request that his modifications be merged back into the original community repository. Using the fork approach also acknowledges the contributions of the original authors, making it a more respectful and transparent process compared to simply downloading the project and uploading it as a new repository under the contributor's account.

After forking, the contributor can clone the project from his GitHub account to the local computer. He can then modify and test the code locally. Once the changes are made, he can upload the updated code to their forked repository using
\begin{lstlisting}
$ git push
\end{lstlisting}

\begin{shortbox}
\Boxhead{Is it possible to push to the original repository directly?}

It is worth mentioning that if \verb|git fork| were not used to fork the repository to the contributor's GitHub account, this \verb|git push| will run into a permission issue. This is because the contributor does not possess ownership to the original repository, hence prevented from modifying it.

It is not recommended, but possible, that the contributor can contact the owner of the repository and ask for permission. The owner of the original repository can invite the contributor as a collaborator of the project. In that case, the contributor will be able to create and push branches to the original repository. Notice that likely the contributor cannot merge his branch to the master branch and still needs pull request to do so.

\end{shortbox}

When the forked repository has been updated with improvements, the contributor can create a pull request in the original community repository. The maintainers of the community project will review the pull request, evaluate the modifications, and either reject it, provide feedback, or approve it for merging.

It is possible that while the contributor is working on their changes, new features or updates are introduced in the original community project. To keep their fork up-to-date, the contributor can add the original repository as a second remote, commonly named ``upstream''. The forked repository under their account is typically referred to as ``origin''. To synchronize with the original repository, the contributor can execute the following commands.
\begin{lstlisting}
$ git fetch upstream
$ git checkout master
$ git rebase upstream/master
\end{lstlisting}

By following this procedure, the contributor ensures that his work remains aligned with the latest developments in the community project, reducing the chances of conflicts and making their contributions easier to integrate.
