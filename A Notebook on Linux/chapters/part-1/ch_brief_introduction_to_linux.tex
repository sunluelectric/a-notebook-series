%\chapterauthor{Author Name}{Author Affiliation}
%\chapterauthor{Second Author}{Second Author Affiliation}
\chapter{Brief Introduction to Linux}

This chapter gives a brief introduction to Linux, including its key features, advantages and disadvantages over other operating systems.

\section{Linux as an Operating System}
\mync{Linux} is one of the most widely used operating systems. An \mync{operating system}[OS] is essentially a special piece of software running on a machine (desktop, laptop, server, mobile devices, edge device, etc.) that manages hardware resources and supports application software in the system. An OS shall be able to handle at least the following tasks.
\begin{itemize}
  \item Detect and prepare hardware
  \item Manage process
  \item Manage memory
  \item Manage filesystem and storage
  \item Provide user interface and \myabb{Application Programming Interface}{API}
  \item Provide \myabb{Software Development Kit}{SDK} for software development
\end{itemize}

Linux has been overwhelmingly successful and adopted in many areas. For example, Android operating system for mobile phones is developed using Linux. Google Chrome is also backed by Linux. Many websites such as Facebook are also running on Linux servers. Some of the most favorable features of Linux, especially to large-size enterprises, are listed below.
\begin{itemize}
  \item Clustering. 
  
  It is possible to group multiple Linux machines and let them work as a whole. The group of machines appears to be a single powerful machine to the upper layer.
  
  \item Virtualization. 
  
  It is possible to share a server among multiple users and applications in a logically separated manner, so that each of the users thinks that they are working on a dedicated machine.
  
  \item Cloud computing. 
  
  Cloud computing is an advanced usage of Linux clustering and virtualization features. Linux servers can be configured flexibly to support cloud computing functions. It is convenient to manage and audit the users and the resources they deploy. 
  
  \item Real-time computing and edge computing. 
  
  Embedded Linux can be implemented on micro-controllers or micro-computers for real-time edge control.
\end{itemize}
This list can go on and on.

Linux, Microsoft Windows and macOS are all successful OSes, yet they differ in many ways. Among the three OSes, Linux is the only completely open-source OS, and can be deployed free-of-charge and customized as requested.

\section{A Brief History of Linux}

The initial motivation of Linux is to create a UNIX-like OS that can be freely distributed in the community.

Many modern OSes including macOS and Linux are inspired by UNIX. UNIX was created by AT\&T in 1969 as a software development environment that it used internally. In 1973, UNIX was rewritten in C language, thus gaining useful features such as portability. Today, C is still the primary language used to create UNIX and Linux kernels.

AT\&T, who originally owned UNIX, tried to make money from it. Back then AT\&T was restricted from selling computers by the government. Therefore, AT\&T decided to license UNIX source code to universities for a nominal fee. Researchers from universities started learning and improving UNIX, which speeded up the development of UNIX. In 1976, UNIX V6 became the first UNIX that was widely spread. UNIX V6 was developed at UC Berkeley and was named the \myabb{Berkeley Software Distribution}{BSD} of UNIX.

From then on, UNIX moved towards two separated directions. While BSD remained ``open'', AT\&T started steering UNIX towards commercialization. By 1984, AT\&T was pretty ready to start selling commercialized UNIX, namely ``AT\&T: \myabb{UNIX System Laboratories}{USL}''. As AT\&T was not allowed to sell PCs to the end users, the only thing it could do was to license the source code to other PC manufactures with a high price. This has prevented end users from obtaining UNIX source code. Although the community acknowledged that UNIX was useful, UNIX source code was extremely costly and was not popular among the end users.

In 1984, Richard Stallman started the GNU project as part of the Free Software Foundation. It is recursively named by phrase ``\myabb{GNU is Not UNIX}{GNU}'', intended to become a recording of the entire UNIX that could be open and freely distributed. The community started to ``recreate'' UNIX based on the defined interface protocols published by AT\&T.

Linus Torvalds started creating his version of UNIX, i.e. Linux, in 1991. He managed to publish the first version of the Linux kernel on August 25, 1991, which only worked on a 386 processor. Later in October, Linux 0.0.2 was released with many parts of the code rewritten in C language, making it more suitable for cross-platform usage. This Linux kernel was the last and the most important piece of code to complete a UNIX-like system under GNU \myabb{General Public License}{GPL}. It is so important that people call it ``Linux OS'' instead of ``GNU OS'', although GNU is the host of the project and Linux kernel is just a part (the most important part) of it.

\section{Linux Distributions}

As casual Linux users, people do not want to understand and compile the Linux source code to use Linux. In response to this need, different Linux distributions have emerged. They often come with corresponding installation packages friendly to amateur users. They share the same Linux OS kernel, but differ from each other in the add-on such as software management tools and default user interfaces.

Today, there are hundreds of Linux distributions in the community. The most famous two categories of distributions are as follows. The major difference is the way they manage software applications.
\begin{itemize}
  \item Red-Hat-Based Distributions
  \begin{itemize}
    \item \myabb{Red Hat Enterprise Linux}{RHEL}
    \item Fedora
    \item CentOS
  \end{itemize}
  \item Debian-Based Distributions
  \begin{itemize}
    \item Debian
    \item Ubuntu
    \item Linux Mint
    \item Elementary OS
    \item Raspberry Pi OS
  \end{itemize}
\end{itemize}

Notice that although the source code of all the distributions above is publicly available as required by GPL (GPL requires that any modified versions of a GPL-licensed product shall also be made open-source with a GPL license, as long as the modifications spread in the community), some of the distributions may come with a ``subscription fee''. The subscription fee is not for the OS source code, but for the technical support, paid maintenance, and other add-on services that the developers of the distributions provide to the end users.

The two types of distributions differ in many aspects, the most important one being the package management methods. More details are introduced as follows.

\subsection{Red-Hat-Based Distributions}

Red Hat created the \mync{Red Hat Package Manager}[RPM] to manage software applications. The RPM packaging contains not only the software files but also its metadata, including version tracking, the creator, the configuration files, etc. In the OS, a local RPM database is used to track all software on the machine. \mync{Yellow Dog Updater Modified}[YUM] is an open-source Linux package management application that uses RPM plus additional features for enhanced user experience. YUM is very popular among Red-Hat-based distributions. As of this writing, YUM is considered legacy and \mync{Dandified YUM}[DNF] is used as its replacement. Compared with YUM, DNF has a vastly improved package dependency resolution, hence is more efficient.

RHEL is a commercial, stable and well-supported OS that can host mission-critical applications for enterprises and governments. To use RHEL, customers pay for subscriptions which allow them to deploy any version of RHEL as desired. Different tiers of supports are available depending on the subscriptions. Many add-on features are available for the customers such as the cloud computing integration.

CentOS is a recreation version of RHEL using freely available RHEL source code. In this sense, CentOS experience should be very similar with RHEL and it is free-of-charge, but the users will not enjoy the professional technical support from RHEL engineers. Recently, Red Hat took over the development of CentOS project.

Fedora is a free, cutting-edge Linux distribution sponsored by Red Hat. It plays as the testbed for Red Hat to interact with the community and test new features. From this perspective, Fedora is very similar to RHEL, just with more dynamics and uncertainties. Some functions, especially server related functions, will be tested on Fedora before implemented on RHEL.

\subsection{Debian-Based Distributions}

Different from Red-Hat-based distributions that use RPM, Debian and Debian-based distributions use \mync{Advanced Packaging Tool}[APT] to manage software applications. APT simplifies the process of managing software by automating the retrieval, configuration and installation of software packages. Among all the Debian-based distributions, Ubuntu is the most successful and popular one. Ubuntu has a variety of graphical tools and focuses on full-featured desktop system while still offering popular server packages. It has a very active community to support its development.

Ubuntu has a larger software pool than Fedora. Ubuntu and its associated software usually have a longer lifespan than Fedora because Ubuntu serves as a stable platform while Fedora is more of a testbed. Ubuntu is more for casual users and beginners, while Fedora more for advanced users or developers, especially developers for RHEL.

More about RPM, YUM, DNF and APT are introduced in Chapter \ref{ch:software-management}.

\section{Linux Graphical Desktop}

Graphical user interface is not necessary to run Linux OS. Yet, many Linux distributions support graphical desktops for the convenience of end users. When installing these distributions, the user has the option to install graphical desktop environments along with the OS. 

The most popular graphical desktop environment is probably GNOME. There are other choices such as KDE, LXDE and Xfce desktops. GNOME and KDE are more for regular PCs while LXDE and Xfce, being light in size, more for low-power-demanding systems. GNOME adopts a more Linux/macOS style desktop environment. KDE, on the other hand, adopts the ``Windows 7'' style. LXDE and Xfce are more simple in graphics presentations since they are more for embedded systems.

It is possible to install multiple desktop environments on one machine, in which case the user can choose which desktop environment to use each time the computer is started.

\section{Linux Installation}

Linux can be installed both on a fixed hard drive or on a mobile storage such as a thumb drive. The installation of different distributions may differ. Thanks to the graphical installation tools for the popular distributions, the installations can be done fairly easily.

Instructions of installing Ubuntu is given by \cite{ubuntu2025ubuntu}. Instructions of installing Fedora is given by \cite{fedora2025fedora}. For the use of RHEL, consult with Red Hat at \cite{rhel2025rhel}. Red Hat provides different types of RHEL licenses for different using purpose, including developer license which is cheaper than a standard enterprise-level license and serves well for learning purpose.
