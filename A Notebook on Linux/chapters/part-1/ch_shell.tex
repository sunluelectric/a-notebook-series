\chapter{Shell} \label{ch:sb}

Linux command Line interface, usually known as the ``shell'' or ``terminal'', is the most available, flexible and powerful tool for the users and programs to interact with the OS and perform actions.

Notice that the shell is not compulsory for casual users when graphical desktop environment is present. Still, it is strongly recommended that the users shall understand at least the basics of the shell since it is more flexible and powerful than the graphical desktop and hence can become handy time to time when configuring certain software.

Linux shell will be used rapidly in the remaining of this notebook.

\section{Shell}

Linux's default \mync{Command Line Interface}[CLI], known as the \mync{shell}, was invented before the graphical tools, and it has been more powerful and flexible than the graphical tools from the first day. On those machines where no graphical desktops are installed, the use of shell is critical.

\subsection{Shell Types}

There are different types of shells for Linux. The most commonly used shell is Bourne Again Shell, also known as \mync{bash}. It is derived from the ``Bourne Shell'' used in UNIX. Unless otherwise mentioned, the shell we refer to in the remaining of the notebook is bash. An example of a shell that calculates Fibonacci series is given below. Notice that in practice shell is mainly used for interaction with the OS and not for software development, data processing or numerical calculation. The example is only for demonstration.

\begin{lstlisting}
#!/usr/bin/bash
n=10
function fib
{
  x=1; y=1
  i=2
  echo "$x"
  echo "$y"
  while [ $i -lt $n ]
  do
      i=`expr $i + 1 `
      z=`expr $x + $y `
      echo "$z"
      x=$y
      y=$z
  done
}
r=`fib $n`
echo "$r"
\end{lstlisting}

Some other shells such as ``C Shell'' and ``Korn Shell'' are also popular among certain users or certain Linux distributions. For example, C Shell supports C-like shell programming, which is sometimes more convenient than bash. In case where a Linux distribution does not come with a particular shell, the user can install it just like installing any other software.

\subsection{Prompt}

With a newly started shell, a string (usually containing username, hostname, current working directory, etc.) followed by either ``\verb|$|'' or ``\verb|#|'' should appear. Following the string, the user can key in shell commands. The string may appear differently on different Linux distributions. An example is given below.
\begin{lstlisting}
<username>@<hostname>:~$
\end{lstlisting}

The above string is called a \mync{prompt}, indicating the start of a user command. By default, for regular users the ending of the prompt is \verb|$|, while for the root user the ending is \verb|#|. The prompt can be customized by changing the environment variable \verb|PS1|. See Sections \ref{ch:sb:subsec:shellenvvar} and \ref{ch:sb:subsec:customizeshell} for details about environment variables and shell customization methods respectively. Commonly seen items are summarized in Table \ref{ch:sb:tab:promptvariable}.

\begin{table}
	\centering \caption{Commonly seen items in shell prompt, and their corresponding variable symbols in \texttt{PS1}.}\label{ch:sb:tab:promptvariable}
	\begin{tabularx}{\textwidth}{lX}
		\hline
		Variable & Description \\ \hline
		\verb|\!| & Command number of the current command in the command history. \\ 
		\verb|\#| & Command number of the current command in the active shell. \\ 
		\verb|\$| & User prompt ``\verb|$|'' for a regular user, or ``\verb|#|'' for the root user. \\
		\verb|\w| & Current working directory entire path. \\ 
		\verb|\W| & Current working directory base name. \\ 
		\verb|\h| & Host name. \\ 
		\verb|\d| & Current date. \\ 
		\verb|\t| & Current time. \\ 
		\verb|\u| & Username. \\ 
		\verb|\s| & Shell name, for example ``\verb|bash|''. \\ 
		\hline
	\end{tabularx}
\end{table}

For the term \mync{root user} or root for short, we are referring to a special user with username and \myabb{User ID}{UID} ``root'' and 0 respectively. This UID gives him the administration privilege over the machine, such as adding/removing users, changing ownership of files, etc. For security purposes, root user shall not be used unless absolutely necessary. For this reason, the root user's authentication is often deactivated by default. This can be done by setting its login password to invalid.

Notice that the root user is different from a \mync{sudoer}, later of which is basically a regular user equipped with \mync{sudo privilege}. A sudoer may temporarily switch to the root user by using \verb|sudo su| as follows.
\begin{lstlisting}
<username>@<hostname>:~$ sudo su
[sudo] password for <username>:
root@<hostname>:/home/<username>#
\end{lstlisting}
More about sudo privilege, \verb|sudo| and \verb|su| commands are introduced in Chapter \ref{ch:usermanagement}.

A Linux shell command looks like the following in general
\begin{lstlisting}
$ <command> [<option>] [<input>]
\end{lstlisting}

The shell comes with built-in commands supported by Linux. Different Linux distributions may support different commands. Shell can also trigger the execution of applications installed in the system. More about commonly used commands and applications will be introduced in the remaining of the notebook.

\subsection{Basic Shell Commands}

A small portion of basic shell commands such as setting alias to commands, adding comments, running commands in the background, etc.,  are introduced here. Commands tied to specific tasks such as file management, workspace navigation, etc., are introduced in their corresponding chapters in the remaining of this notebook.

When login to an unfamiliar system, the first step is often to check the basic system information, such as hardware configuration, OS version, username, hostname, etc. The following commands show basic information of a user.
\begin{lstlisting}
$ whoami
<username>
$ grep <username> /etc/passwd
<username>:x:<uid>:<gid>:<gecos>:<home-directory>:<shell>
\end{lstlisting}
In the above, \verb|whoami| is used to display the current login user's username. Command \verb|grep| is used to search content from files or folders, in this case \verb|<username>| from \verb|/etc/passwd| file. The \verb|/etc/passwd| file stores basic user information. This should return the username, the password (for encrypted password, an ``x'' is returned), UID, \myabb{Group ID}{GID}, user id info, home directory and default shell location of the user.

The following commands show the date and hostname of the machine.
\begin{lstlisting}
$ date
<date, time and timezone>
$ hostname
<hostname>
\end{lstlisting}

The following command \verb|lshw| lists down hardware information in details. Notice that sudo privilege is recommended when using this command, to give more detailed and accurate information of the system. Sometimes displayed information can be too detailed. Use \verb|-short| argument with the command to shorten the output.
\begin{lstlisting}
$ sudo lshw
\end{lstlisting}

Command \verb|alias| is used to create short-cut keys for commands and associated options, which makes it more convenient for the system operators to work on the shell. Some alias has already been created automatically when the shell is started. Use \verb|alias| to check the existing alias in the shell. An example is given below.

\begin{lstlisting}
$ alias
alias egrep='egrep --color=auto'
alias fgrep='fgrep --color=auto'
alias grep='grep --color=auto'
alias l='ls -CF'
alias la='ls -A'
alias ll='ls -alF'
alias ls='ls --color=auto'
\end{lstlisting}

A temporary alias can be added to the shell by using
\begin{lstlisting}
$ alias <shortcut command>='<original command and options>'
\end{lstlisting}
for example
\begin{lstlisting}
$ alias pwd='pwd; ls -CF'
\end{lstlisting}

To permanently add alias to the shell, the alias needs to be added to \verb|~/.bashrc|, which is a piece of shell script that is automatically executed each time a new shell is started by the user. It is sometimes known as the user script.

Some commands are powerful yet difficult to use. Consider using the following two methods to check the user manual about a command.
\begin{lstlisting}
$ man <command>
$ <command> --help
\end{lstlisting}

Use \verb|#| to lead a single-line comment. For multi-line comment, use \verb|:' ... '|. Examples are given below.
\begin{lstlisting}
# this is a single-line comment
:'
this is a multi-line comment
this is a multi-line comment
this is a multi-line comment
'
\end{lstlisting}

Use \verb|history| to check history commands. Use \verb|!<history command index>| to repeat a history command, or use \verb|!!| to repeat the latest previous command. It is possible to disable history recording function for privacy purpose.

Use \verb|\| to break a single-line command into multiple rows when the command is very long.

\subsection{Command Chain}

A truly power feature of the shell is its ability to redirect inputs and outputs of commands, thus to chain the commands together. Meta-characters pipe (\verb?|?), ampersand (\verb|&|), semicolon (\verb|;|), dollar (\verb|$|), parenthesis (\verb|()|), square bracket (\verb|[]|), less than sign (\verb|<|), greater than sign (\verb|>|), double greater than sign (\verb|>>|) and error greater than sign (\verb|2>|) are commonly used special characters in shell commands that directs the outputs of commands. Details are given below.

The pipe (\verb$|$) connects the output of the first command to the input of the second command. The following example searches keyword ``function'' in \verb|calculate_fib.sh| which was given previously.
\begin{lstlisting}
$ cat calculate_fib.sh | grep function
function fib
\end{lstlisting}
where \verb|cat| concatenates files and print on the standard output, and \verb|grep| prints lines that match patterns in each file.

The semicolon (\verb|;|) allows inputting multiple commands in the same line in the script. The commands are then executed one after another from left to right.

The ampersand (\verb|&|) can be put in the end of a line so that the command on that line will run in the background. The commands or process running in the background does not occupy the shell standard display, and the users can continue working on other commands in parallel. This is particularly useful when a task is going to take a long time to be executed. To manage the tasks running in the background, check more details in Chapter \ref{ch:pm}.

Use the dollar sign \verb|$| (not the prompt) to indicate a command expansion. The command in \verb|$(<command>)| will be executed as a whole, then treated as a single argument. The content in \verb|()| is sometimes called sub-shell. For example, to count the number of files/folders in the current directory, use
\begin{lstlisting}
$ echo There are $(ls -a | wc -w) files in this directory.
There are 69 files in this directory.
\end{lstlisting}
where \verb|echo|, as its name suggests, is used to repeat its input and put it into visible output, and \verb|wc| counts the number of lines, words or bytes in a file.

\subsection{Basic Calculations}

Use \verb|$[<arithmetic expression>]| for simple calculations, such as
\begin{lstlisting}
$ echo 1+1=$[1+1]
1+1=2
\end{lstlisting}

The dollar sign \verb|$| is also used to expand the value of a variable, either environmental variable or self-defined variable, as explained previously in \ref{ch:sb:subsec:shellenvvar}. An example is \verb|$PATH| which returns the PATH environment.

The less than sign \verb|<| and greater than sign \verb|>| are used to reroute the input and output of a command to a file instead of the standard input and output. An example using command \verb|sort| together with input direction \verb|<| is given as follows. Considering sorting characters ``a'', ``c'', ``b'', ``g'', ``e'', ``f'', ``d'' using \verb|sort| command. The letters are input from the console as follows. Use \verb|ctrl+D| to quit the input, and the output after sorting will be displayed in the console as follows.
\begin{lstlisting}
$ sort
a
c
b
g
e
f
d
a
b
c
d
e
f
g
\end{lstlisting}
For demonstration purpose, create a file \verb|before_sort| in the current working directory. Inside \verb|before_sort| are letters ``a'', ``c'', ``b'', ``g'', ``e'', ``f'', ``d'', each occupying a separate row. This can be done by 
\begin{lstlisting}
$ echo -e "a\nc\nb\ng\ne\nf\nd\n" > before_sort
\end{lstlisting}
Use \verb|cat| to quickly check its content as follows.
\begin{lstlisting}
$ cat before_sort
a
c
b
g
e
f
d
\end{lstlisting}
Use \verb|sort| to sort \verb|before_sort| as follows. In this case, the input to \verb|sort| becomes a file, rather than the standard input from the keyboard. Notice that in this example, \verb|sort before_sort| also works, as \verb|sort| will by default take its first argument as the location of the file to be sorted.
\begin{lstlisting}
$ sort < before_sort
a
b
c
d
e
f
g
\end{lstlisting}
Use \verb|>| to redirect the output of a command to a file as given in the following example.
\begin{lstlisting}
$ sort < before_sort > after_sort
$ cat after_sort
a
b
c
d
e
f
g
\end{lstlisting}
where \verb|sort| does not output the result to the console, but instead saves the result in a file named \verb|after_sort|. The double greater sign \verb|>>| works similarly with \verb|>| except that \verb|>>| will append the output to an existing file, while \verb|>| overwrites the existing file.

The error greater sign \verb|2>|, \verb|2>>| works similarly with \verb|>|, \verb|>>| except that instead of redirecting standard output messages, it redirects the error messages.

\section{Shell Environment Variables}\label{ch:sb:subsec:shellenvvar}

\mync{Shell Environmental Variables} are either system-defined or user-defined variables that the shell can access and keep track. Commonly used environmental variables are introduced below.

Variable PATH is one of the most important shell environmental variables. To execute a command by its name, the OS needs to know where the command is located at. Commonly used commands shall be included in PATH, an environment variable of the shell that traces the location of the built-in commands, so that they can be executed anytime from any working directory. The PATH environment variable is a collection of directories in the system, and it is initialized automatically when the shell is started. Check PATH by
\begin{lstlisting}
$ echo $PATH
<directory-1>:<directory-2>:<directory-3>: ...
\end{lstlisting}
where \verb|$PATH| gives the PATH environment of the current shell. The dollar sign \verb|$| is used to retrieve the content of an environment variable. 

The default PATH environment variable often contains the following directories. 
\begin{itemize}
\item \verb|/bin|, \verb|/usr/bin|: Commonly used Linux built-in commands.
\item \verb|/sbin|, \verb|/usr/sbin|: Commonly used Linux built-in commands for administrators.
\item \verb|/home/<username>/bin|: Commands defined by a user.
\end{itemize}
To determine the location of a particular command, use \verb|type| if the command can be found in the PATH environment. Alternatively, use \verb|locate|, \verb|mlocate| to search all the accessible files in the system to find a command. The syntax is demonstrated below.
\begin{lstlisting}
$ type <command>
<command-location>
\end{lstlisting}

There are many useful environment variables apart from \verb|PATH|. A summary is given in Table \ref{ch:sb:tab:shellenvironmentvars}.Command \verb|echo $<variable name>| can be used to check the values of these variables. Use command \verb|env| to check a list of environment variables in the shell.

\begin{table}
	\centering \caption{Commonly used shell environment variables.}\label{ch:sb:tab:shellenvironmentvars}
	\begin{tabularx}{\textwidth}{lX}
		\hline
		Variable & Description \\ \hline
		\verb|BASH| & Full pathname of the \verb|bash| command. \\ 
		\verb|BASH_VERSION| & Current version of the \verb|bash| command. \\ 
		\verb|EUID| & Effective user ID number of the current user, which is assigned when the shell starts, based on the user's entry in \verb|/etc/passwd|. \\ 
		\verb|HISTFILE| & Location of the history file. \\ 
		\verb|HISTFILESIZE| & Maximum number of history entries. \\ 
		\verb|HISTCMD| & The number index of the current command. \\ 
		\verb|HOME| & Home directory of the current user. \\ 
		\verb|PATH| & Path to available commands. \\ 
		\verb|PWD| & Current directory. \\ 
		\verb|OLDPWD| & Previous directory. \\ 
		\verb|SECONDS| & Number of seconds since the shell starts. \\ 
		\verb|RANDOM| & Generating a random number between 0 and 99999. \\
		\hline
	\end{tabularx}
\end{table}

The environmental variables can be created or updated as follows.
\begin{lstlisting}
<variable name> = <variable value> ; export <variable name>
\end{lstlisting}
For example,
\begin{lstlisting}
PATH = $PATH:/getstuff/bin ; export PATH
\end{lstlisting}
adds a new directory \verb|/getstuff/bin| to the \verb|PATH| environmental variable. This allows temporary change to the PATH environment variable. Notice that each time the shell is restarted, the environmental variables are also reset. 

To permanently change the value of an environmental variable, consider customizing the shell. Shell customization is introduced in Section \ref{ch:sb:subsec:customizeshell}.

\section{Customization of Shell} \label{ch:sb:subsec:customizeshell}

Shell configuration files are loaded each time a new shell starts. User-defined permanent configurations (such as useful alias) can be put into these files so that the configurations can be implemented automatically. Some useful files are summarized in Table \ref{ch:sb:tab:shellconfig}.

\begin{table}
	\centering \caption{Shell configuration files.}\label{ch:sb:tab:shellconfig}
	\begin{tabularx}{\textwidth}{lX}
		\hline
		File pathname & Description \\ \hline
		\verb|/etc/profile| & The environment information for every user, which executes upon any user logs in. Root privilege is required to edit this file.  \\ 
		\verb|/etc/bashrc| & Bash configuration for every user, which executes upon any user starts a shell. Root privilege is required to edit this file. \\ 
		\verb|~/.bash_profile| & The environment information for current user, which executes upon the user logs in. \\ 
		\verb|~/.bashrc| & Bash configuration for current user, which executes upon the user starts a shell. \\ 
		\verb|~/.bash_logout| & Bash log out configuration for current user, which executes upon the user logs out or exit the last bash shell. \\ \hline
	\end{tabularx}
\end{table}

To customize the shell behavior for a user, the most commonly method is to edit his \verb|~/.bashrc| which is executed automatically each time he starts a shell. The file is human-readable and self-explanatory.

\section{Basic Shell Script Programming}

We have been discussing shell commands so far. This section briefly introduces shell script programming, which is a collection of shell commands to serve a specific purpose. Shell script can be used both as a standalone commands pipeline and a function that takes into arguments, processes them, and generates results.

\subsection{Typical Shell Script}

A typical shell script looks like the following
\begin{lstlisting}
#!/bin/bash

# <comment>

<VARIABLE>=<value>

<main body>
\end{lstlisting}
where
\begin{itemize}
  \item \verb|#!/bin/bash| is known as the \mync{shebang}, which tells the shell what program to interpret the script with. In this example, \verb|/bin/bash| is used as the interpreter of the script.
  \item \verb|# <comment>| gives a brief explanation to the script.
  \item \verb|<VARIABLE>=<value>| assigns global variables of the script.
\end{itemize}

\subsection{Conditional Statements and Loops}

Conditional statements in bash shell script include \verb|if| and \verb|case|, the syntax of which is given below.

\begin{lstlisting}
if [<statement>]
then
  <body>
elif [<statement>]
then
  <body>
else
  <body>
fi
\end{lstlisting} 

\begin{lstlisting}
case <expression> in
  <pattern> )
    <body>
    ;;
  <pattern> | <pattern> | <pattern> )
    <body>
    ;;
  *)
    <body>
    ;;
esac
\end{lstlisting}

Notice that the syntax of \verb|case| in bash shell script looks different from the most other programming languages such as \verb|C| or \verb|Java|. For example, it uses single right bracket \verb|)| to separate the case pattern and body instead of \verb|{}| or \verb|:|, and \verb|;;| to terminate a body instead of \verb|break|. This is because the syntax evolved from the older Unix shell conventions.

Commonly used loops include \verb|while| and \verb|for|. The syntax is given below.

\begin{lstlisting}
while [<statement>]
do
  <body>
done
\end{lstlisting}

\begin{lstlisting}
for <item> in <item 1> <item 2> <item 3> # general syntax
do
  <body>
done
\end{lstlisting}

\begin{lstlisting}
for <index> in {<start>..<end>..<interval>} # number ranges
do
  <body>
done
\end{lstlisting}

The \verb|for| loop can also be used to navigate files in a directory. For example,
\begin{lstlisting}
for <file> in <regular expression>
do
  <body>
done
\end{lstlisting}
loops over all the files in the current directory, with the regular expression used to filter the applicable files.



















