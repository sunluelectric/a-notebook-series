%\chapterauthor{Author Name}{Author Affiliation}
%\chapterauthor{Second Author}{Second Author Affiliation}
\chapter{Administration} \label{ch:administration}
...

\section{Introduction to Linux Administration}
...


\section{Root Account Management}

Managing and protecting the \textit{root} user account is a key portion in Linux administration. It is worth mentioning that the root user is different from a ``\textit{sudoer}'', i.e., a user that can use \verb|sudo| command, although they both have elevated privileges when comes to system administration. A more detailed explanation is given below.

The root user refers to the user that is created by the system upon first installation of the OS. Its username is by default ``root'', with a UID of 0, and a GID of 0. Its home directory is by default \verb|/root| instead of \verb|/home/<user name>|. The default prompt for the root user is often \verb|#| instead of \verb|$| for other users. The root user has administrative privileges and can do almost anything without being denied or questioned by the system. As a commonly used safety practice, the password for the root user is usually disabled, thus nobody can login to the system as the root user using username and password authentication.

Notice that in theory the root user does not necessarily need to use the username ``root'', although it is a default convention. The administrative  comes with the UID of 0, not the username. Thus, it is possible to assign the administrative account a different username. It is also possible to create multiple accounts with administrative privileges by assigning UID of 0 to multiple accounts, although it is not recommended to share the same UID among different users.

Although the administrative privilege of root user does not come with its username, in some systems, the username ``root'' is given elevated privilege anyway. More details about this is introduced later.

Check the root user information as follows.
\begin{lstlisting}
$ cat /etc/passwd | grep ^root
root:x:0:0:root:/root:/bin/bash
\end{lstlisting}
where the third and fourth column of above give the UID and GID of the root user, respectively.

For comparison, a regular user would have far different UID and GID as shown below.
\begin{lstlisting}
$ cat /etc/passwd | grep ^sunlu
sunlu:x:1000:1000:Sun Lu,,,:/home/sunlu:/bin/bash
\end{lstlisting}

A sudoer refers to the regular users who can temporarily elevate its privileges and execute administration commands using \verb|sudo <privileged-command>|. A sudoer can also switch to root user prompt using \verb|sudo su| (and quit root user prompt using \verb|exit|). Their sudo privilege comes from the fact that they are included in the ``sudo group''.

Use \verb|groups| to check existing defined groups in the system, and \verb|groups <user name>| the groups a user is engaged. An example is given below.
\begin{lstlisting}
$ groups
sunlu adm cdrom sudo dip plugdev lpadmin lxd sambashare docker
$ groups sunlu
sunlu : sunlu adm cdrom sudo dip plugdev lpadmin lxd sambashare docker
\end{lstlisting}

The elevated privilege of the sudoer group is defined in \verb|/etc/sudoers|. A section of the file is given below, where \verb|%| appeared in front of \verb|admin| and \verb|sudo| is used to indicate group name.
\begin{lstlisting}
# User privilege specification
root	ALL=(ALL:ALL) ALL

# Members of the admin group may gain root privileges
%admin ALL=(ALL) ALL

# Allow members of group sudo to execute any command
%sudo	ALL=(ALL:ALL) ALL
\end{lstlisting}

It is possible to give the root account a password, thus enabling root authentication. This approach is sometimes used for troubleshooting and recovering purposes, but it is not a good practice in routine operations.

\section{User Management}

xxx
















