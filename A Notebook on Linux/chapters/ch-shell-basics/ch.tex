%\chapterauthor{Author Name}{Author Affiliation}
%\chapterauthor{Second Author}{Second Author Affiliation}
\chapter{Shell Basics} \label{ch:sb}

Linux command line tool, usually known as the ``shell'', is the most powerful tool for Linux operations including configuration and control of the OS. Notice that the use of the shell is not compulsory for casual users when the graphical desktop is present. Though the shell is not as intuitive as the graphical tools, it is more powerful and flexible, and well-supported by the community.

Linux shell will be used rapidly in the remaining sections of this notebook.

\section{Brief Introduction}

Linux command line tool, usually known as Linux shell, was invented before the graphical tools, and it has been more powerful and flexible than the graphical tools from the first day. On those machines where no graphical desktops are installed, the use of shell is critical.

\subsection{Shell Types}

There are different types of shells. The most commonly used shell is the ``bash shell'' which stands for ``Bourne Again Shell'', derived from the ``Bourne Shell'' used in UNIX. An example \verb|calculate_fib.sh| written in bash shell script is given below, where the first 10 terms in Fibonacci series is calculated ``1, 1, 2, 3, 4, 8, 13, 21, 34, 55''.

\begin{lstlisting}
#!/usr/bin/bash
n=10
function fib
{
  x=1; y=1
  i=2
  echo "$x"
  echo "$y"
  while [ $i -lt $n ]
  do
      i=`expr $i + 1 `
      z=`expr $x + $y `
      echo "$z"
      x=$y
      y=$z
  done
}
r=`fib $n`
echo "$r"
\end{lstlisting}

Some other shells such as ``C Shell'' and ``Korn Shell'' are also popular among certain users or certain Linux distributions. For example, C Shell supports C-like shell programming, which can sometimes be more convenient than the bash shell. In case where the Linux distribution does not have these shells pre-installed, the user can install and use these shells just like installing other software.

In this notebook, bash shell is assumed unless otherwise mentioned.

\subsection{Prompt and Basic Concepts}

After opening the shell or terminal, you will see a string (usually containing username, hostname, current working directory, etc.) followed by either a \verb|$| or \verb|#|, starting from where you can input your shell command. For example, it may look like the following:
\begin{lstlisting}
<username>@<hostname>:~$
\end{lstlisting}

The above displayed string is called a \textit{prompt}, indicating the start of a manually input command. By default, for regular user, the ending of the prompt is \verb|$| while for the root user, the ending is \verb|#|. The prompt can be customized by changing the environment variable \verb|PS1|. See Sections \ref{ch:sb:subsec:shellenvvar}, \ref{ch:sb:subsec:others} for details about environment variable and shell configuration, respectively.

For the term ``root user'', we are referring to a special user with username and user ID (UID) ``root'' and 0 respectively. This UID gives him the administration privilege over the machine, such as adding/removing users, change ownership of files, etc. To avoid vital damage by human error, root user shall not be used unless it is definitely necessary. For this reason, the root user's authentication is often deactivated by default (by setting its login password to invalid).

Notice that the root user is different from a ``sudoer'', later of which is basically a regular user equipped with \textit{sudo privilege}. However, a sudoer can temporarily switch to root user by using \verb|sudo su| as follows.
\begin{lstlisting}
<username>@<hostname>:~$ sudo su
[sudo] password for <username>:
root@<hostname>:/home/<username>#
\end{lstlisting}

More about sudo privilege, \verb|sudo| and \verb|su| commands are introduced later parts of the notebook.

You can key in a command after the prompt, and execute the command by pressing the \verb|Enter| key. A Linux shell command usually has the following form.
\begin{lstlisting}
$ <command> <option> <input>
\end{lstlisting}

\section{Useful Commands}

Some useful commands are introduced in this section by categories.

Notice that text file editing, files management (such as changing ownership of files), software management (such as installation of software, checking version, upgrading software, etc.), process management (such as checking CPU usage, terminating a process) are commonly used in Linux operations. They are not included in this chapter, but introduced separately in later chapters.

\subsection{Set Shell Environment Variables}\label{ch:sb:subsec:shellenvvar}

Unless the directory where a command resides is specifically given, the OS only recognizes the commands saved in the PATH environment of the shell. The PATH environment is a series of directories (locations) in the system, and it is initialized automatically when the shell is started. Check the PATH environment by
\begin{lstlisting}
$ echo $PATH
<directory-1>:<directory-2>:<directory-3>: ...
\end{lstlisting}
where \verb|echo| displays a line of text, and \verb|$PATH| is a built-in enviornmental variable that records the PATH environment of the current bash. It is possible to include new directories to PATH environment either temporarily or permanently to integrate more commands.

Most Linux-defined user commands are stored under \verb|/bin|, \verb|/usr/bin|, and administrative commands in \verb|/sbin|, \verb|/usr/sbin|. Commands local to a specific user can be stored under \verb|/home/<username>/bin|. To determine the location of a particular command, use \verb|type| if the command is in \verb|$PATH|, or \verb|locate|, \verb|mlocate| to search all the accessible files in the system. An example is given below.
\begin{lstlisting}
$ type <command>
<command-location>
\end{lstlisting}

In addition to \verb|PATH|, there is a list of other shell environment variables for the user to monitor and control and status of the system. Table \ref{ch:sb:tab:shellenvironmentvars} summarizes commonly used shell environment variables. Command \verb|echo $<variable-name>| can be used to check the values of these variables.

\begin{table}
	\centering \caption{Commonly used shell environment variables.}\label{ch:sb:tab:shellenvironmentvars}
	\begin{tabularx}{\textwidth}{lX}
		\hline
		Variable & Description \\ \hline
		\verb|BASH| & Full pathname of the \verb|bash| command. \\ \hdashline
		\verb|BASH_VERSION| & Current version of the \verb|bash| command. \\ \hdashline
		\verb|EUID| & Effective user ID number of the current user, which is assigned when the shell starts, based on the user's entry in \verb|/etc/passwd|. \\ \hdashline
		\verb|HISTFILE| & Location of the history file. \\ \hdashline
		\verb|HISTFILESIZE| & Maximum number of history entries. \\ \hdashline
		\verb|HISTCMD| & The number index of the current command. \\ \hdashline
		\verb|HOME| & Home directory of the current user. \\ \hdashline
		\verb|PATH| & Path to available commands. \\ \hdashline
		\verb|PWD| & Current directory. \\ \hdashline
		\verb|OLDPWD| & Previous directory. \\ \hdashline
		\verb|SECONDS| & Number of seconds since the shell starts. \\ \hdashline
		\verb|RANDOM| & Generating a random number between 0 and 99999. \\
		\hline
	\end{tabularx}
\end{table}

The environmental variables can be edited and new environmental variables can be created as follows.
\begin{lstlisting}
<variable-name> = <variable-value> ; export <variable-name>
\end{lstlisting}

For example,
\begin{lstlisting}
PATH = $PATH:/getstuff/bin ; export PATH
\end{lstlisting}
adds a new directory \verb|/getstuff/bin| to the \verb|PATH| environmental variable.

Use command \verb|env| to check a list of environment variables in the shell.

\subsection{Display User Information}

Administrative users may need to frequently check the basic system information, such as hardware configuration, OS version, username, hostname, disk usage, running process, system clock, etc. Some useful commands are summarized below.

The following commands show basic information of a user.
\begin{lstlisting}
$ whoami
<username>
$ grep <username> /etc/passwd
<username>:x:<uid>:<gid>:<gecos>:<home-directory>:<shell>
\end{lstlisting}
In the above, \verb|whoami| is used to display the current login user's username. Command \verb|grep| is used to search a content (in this case, the user name) in the selected file \verb|/etc/passwd| where the user information is stored. This should return the username, the password (for encrypted password, an ``x'' is returned), UID, group id (GID), user id info (GECOS), home directory and default shell location of the user.
Another command \verb|id| also returns the UID and GID information of the current user.

\subsection{Display Machine Information}

The following commands show the date and hostname of the machine.
\begin{lstlisting}
$ date
<date, time and timezone>
$ hostname
<hostname>
\end{lstlisting}

The following command \verb|lshw| lists down hardware information in details. Sudo privilege is recommended when using this command, to give detailed and accurate information of the system. Sometimes the displayed information can be too detailed, thus it is more convenient to use \verb|-short| argument.
\begin{lstlisting}
$ sudo lshw
\end{lstlisting}

\subsection{Perform Simple Files Operations}

The most important commands for navigating in the file system is to display the current working directory (may be included as part of prompt) and list down files and directories in the current working directory as follows.
\begin{lstlisting}
$ pwd
<working-directory-path>
$ ls
<file-list>
\end{lstlisting}

The aforementioned \verb|ls| command can be used flexibly. Commonly seen arguments that come with \verb|ls| are \verb|-l| (implement long listing with more details of each item), \verb|-a| (include hidden item in the list) and \verb|-t| (list by time).

More file operations related commands are introduced in Chapter \ref{ch:fm}.

\subsection{Set Alias and Shortcuts}

Command \verb|alias| is used to create short-cut keys for commands and associated options, which makes it more convenient for the system operators to work on the shell. Some alias has already been created automatically when the shell is started. Use \verb|alias| to check the existing alias in the shell. An example is given below.

\begin{lstlisting}
$ alias
alias egrep='egrep --color=auto'
alias fgrep='fgrep --color=auto'
alias grep='grep --color=auto'
alias l='ls -CF'
alias la='ls -A'
alias ll='ls -alF'
alias ls='ls --color=auto'
\end{lstlisting}

A temporary alias can be added to the shell by using
\begin{lstlisting}
$ alias <shortcut command>='<original command and options>'
\end{lstlisting}
for example
\begin{lstlisting}
$ alias pwd='pwd; ls -CF'
\end{lstlisting}

To permanently add alias to the shell, the alias needs to be added to the bash start script, which is usually \verb|~/.bashrc| for a user. See Section \ref{ch:sb:subsec:others} for details.

\subsection{Others}\label{ch:sb:subsec:others}

Many commands can be used flexibly and it is impossible to illustrate all their details. Consider use the following two methods to check the detailed manual about a command.
\begin{lstlisting}
$ man <command>
$ <command> --help
\end{lstlisting}

Use \verb|history| to check history commands. Use \verb|!<history command index>| to repeat a history command, or use \verb|!!| to repeat the latest previous command. It is possible to disable history recording function for privacy purpose.

Shell configuration files are loaded each time a new shell starts. User-defined permanent configurations (such as useful alias) can be put into these files so that the configurations can be implemented automatically. Some useful files are summarized in Table \ref{ch:sb:tab:shellconfig}.

\begin{table}
	\centering \caption{Shell configuration files.}\label{ch:sb:tab:shellconfig}
	\begin{tabularx}{\textwidth}{lX}
		\hline
		File pathname & Description \\ \hline
		\verb|/etc/profile| & The environment information for every user, which executes upon any user logs in. Root privilege is required to edit this file.  \\ \hdashline
		\verb|/etc/bashrc| & Bash configuration for every user, which executes upon any user starts a shell. Root privilege is required to edit this file. \\ \hdashline
		\verb|~/.bash_profile| & The environment information for current user, which executes upon the user logs in. \\ \hdashline
		\verb|~/.bashrc| & Bash configuration for current user, which executes upon the user starts a shell. \\ \hdashline
		\verb|~/.bash_logout| & Bash log out configuration for current user, which executes upon the user logs out or exit the last bash shell. \\ \hline
	\end{tabularx}
\end{table}

\section{A Taste of Bash Shell Script Programming}

A truly power feature of the shell is its ability to redirect inputs/outputs of commands, thus to chain the commands together. Meta-characters pipe (\verb?|?), ampersand (\verb|&|), semicolon (\verb|;|), dollar (\verb|$|), parenthesis (\verb|()|), square bracket (\verb|[]|), less than sign (\verb|<|), greater than sign (\verb|>|), double greater than sign (\verb|>>|), error greater than sign (\verb|2>|) and a few more more, are used for this feature. Details are given below.

The pipe (\verb$|$) connects the output of the first command to the input of the second command. The following example searches keyword ``function'' in \verb|calculate_fib.sh| which was given previously.
\begin{lstlisting}
$ cat calculate_fib.sh | grep function
function fib
\end{lstlisting}
where \verb|cat| concatenates files and print on the standard output, and \verb|grep| prints lines that match patterns in each file.

The semicolon (\verb|;|) allows inputting multiple commands in the same line in the script. The commands are then executed one after another from left to right.

The ampersand (\verb|&|) can be put in the end of a line so that the command on that line will run in the background. The commands or process running in the background does not occupy the shell standard display, and the users can continue working on other commands in parallel. This is particularly useful when a task is going to take a long time to be executed. To manage the tasks running in the background, check more details in Chapter \ref{ch:pm}.

Use the dollar sign \verb|$| (not the prompt) to indicate a command expansion. The command in \verb|$(<command>)| will be executed as a whole, then treated as a single input. The content in \verb|()| is sometimes called sub-shell. For example, to display the function defined in \verb|calculate_fib.sh| previously,
\begin{lstlisting}
$ echo Display functions: $(cat calculate_fib.sh | grep function)
Display functions: function fib
\end{lstlisting}
Use \verb|$[<arithmetic expression>]| for simple calculations, such as
\begin{lstlisting}
$ echo 1+1=$[1+1]
1+1=2
\end{lstlisting}
Another example to count the number of files/folders in the current directory is
\begin{lstlisting}
$ echo There are $(ls -a | wc -w) files in this directory.
There are 69 files in this directory.
\end{lstlisting}
where \verb|wc| counts the number of lines, words or bytes in a file.

The dollar sign \verb|$| is also used to expand the value of a variable, either environmental variable or self-defined variable, as explained previously in \ref{ch:sb:subsec:shellenvvar}.

The less than sign \verb|<| and greater than sign \verb|>| are used for input/output direction of a file. They are useful when a command needs to pull input and/or push output to a file instead of the standard input and output. An example using command \verb|sort| together with input direction \verb|<| is given as follows. Considering sorting characters ``a'', ``c'', ``b'', ``g'', ``e'', ``f'', ``d'' using \verb|sort| command. The letters are input from the console as follows. Use \verb|ctrl+D| to quit the input, and the output after sorting will be displayed in the console as follows.
\begin{lstlisting}
$ sort
a
c
b
g
e
f
d
a
b
c
d
e
f
g
\end{lstlisting}
For demonstration purpose, create a file \verb|before_sort| in the current working directory. Inside \verb|before_sort| are letters ``a'', ``c'', ``b'', ``g'', ``e'', ``f'', ``d'', each occupying a separate row. There are several ways to create the file, which will be explained later. For now, just assume that the file already exists. Use \verb|cat| to quickly check its content as follows.
\begin{lstlisting}
$ cat before_sort
a
c
b
g
e
f
d
\end{lstlisting}

Use \verb|sort| to sort \verb|before_sort| as follows. In this case, the input to \verb|sort| becomes a file, rather than the standard input from the keyboard. Notice that in this example, \verb|sort before_sort| also works, as \verb|sort| will by default take its first argument as the location of the file to be sorted.
\begin{lstlisting}
$ sort < before_sort
a
b
c
d
e
f
g
\end{lstlisting}

Use \verb|>| to redirect the output of a command to a file as given in the following example.
\begin{lstlisting}
$ sort < before_sort > after_sort
$ cat after_sort
a
b
c
d
e
f
g
\end{lstlisting}
where \verb|sort| does not output the result to the console, but instead saves the result in a file named \verb|after_sort|. The double greater sign \verb|>>| works similarly with \verb|>| except that \verb|>>| will append the output to an existing file, while \verb|>| overwrites the existing file.

With the above been said, it is possible to use the following to create the \verb|before_sort| file that has been used in the example.
\begin{lstlisting}
$ echo -e "a\nc\nb\ng\ne\nf\nd\n" > before_sort
\end{lstlisting}

The error greater sign \verb|2>|, \verb|2>>| works similarly with \verb|>|, \verb|>>| except that instead of redirecting standard output messages, it redirects the error messages.




