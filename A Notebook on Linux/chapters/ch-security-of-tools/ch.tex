\chapter{Services and Applications Security Mechanisms}

The previous chapter introduced general OS security mechanisms. This chapter discusses security mechanisms of services and software running on the OS, including networking, database, etc.

\section{Networking Security}

\section{Database Security}

Many Apps, both local and online, heavily rely on database to manage information. For example, online shopping App uses database to record transaction history. Online banking uses database to store and manage customer capitals. The data stored in a database is often critical and confidential, and the database service providers need to try their best to prevent data loss and leaking, and to ensure the integrity and availability of the database.

When a database is under attack, the worst scenario may go beyond data damage. Examples are given below.
\begin{itemize}
	\item Paralyze database service.
	\item Change and remove data illegally.
	\item Steel data.
	\item Attack and gain unauthorized access to the underlying OS, and damage or control the entire server.
	\item Deploy Trojan horse program for other servers connected to the database server.
\end{itemize}

It is possible that the attacker hides and disguises the attacking command into SQL injection to open a back door or to retrieve unauthorized data such as confidential information.

\begin{shortbox}
\Boxhead{Secure DBMS and Secure OS}
It is worth introducing the relationship between a security-enhanced OS and security-enhanced DBMS. In short, they make a better each other, forming a ``security chain'' together. Sometimes it is possible to use SELinux with a normal DBMS to form a security system (this is what was used in the early days), and with a security-enhanced DBMS to form a very secure system. A poor OS, on the other hand, harms the DBMS security level because data is essentially stored on hard drive which can be penetrated if the OS is down.

There are two common ways of forming a security-enhanced DBMS from a normal DBMS. The first way is to upgrade the DBMS kernel codes for additional security features. This can be complicated and requires a high-level mathematics, databases and programming skills, but can provide a safe DBMS. Obviously, this applies to only open-source databases. The second way is to build a ``wrapper'' for the DBMS to interface with the users and API calls This usually requires less skill sets and can be applied to both open-source and proprietary databases, but can only provide mediocre security.

Security-enhanced DBMS shall have many safety relevant features and regulations. Only a small part of them is introduced in this notebook.
\end{shortbox}

\subsection{Database Security Risks Categories}

Depending on the identity and the access level of the attacker, database security risks can be divided into the following categories as shown in Table \ref{ch:securityaccessories:tab:dbsecurityriskscategories}.
\begin{table}
	\centering \caption{DB security risks categories and associated security methods.} \label{ch:securityaccessories:tab:dbsecurityriskscategories}
	\begin{tabular}{|c|c|c|}
		\hline
			& With Authentication & Without Authentication \\ \hline
		\multirow{2}{*}{Internal} & Developer, manager, etc. & Irrelevant employee, etc. \\
		& Access control and audit & Encoding \\ \hline
		\multirow{2}{*}{External} & Customer, vendor, etc. & Hacker, visitor, etc. \\
		& Outsourced DB security methods & SQL injection prevention \\
		\hline
	\end{tabular}
\end{table}

Different database security risks categories may tell completely different stories on how an attacker plans his attack. For example, a developer may leave a backdoor program in the program and if nobody else spotted the backdoor program, he can use it to access confidential data. A vendor may be able to bring the hard drive of the database outside the managed premises, after which he can use variety of tools to crack the database and obtain the data.

To tackle the challenges, different security methods need to be applied to prevent each and every risks category. A high-level summary is given in Table \ref{ch:securityaccessories:tab:dbsecurityriskscategories}. Details are given in later sections.

\subsection{Database Access Control}

There are different types of access control, some of which widely adopted to all different types of databases, while others may apply to only high-level secure databases. Some access control schemas apply to both database and OS.

\vspace{0.1in}
\noindent \textbf{Discretionary Access Control (DAC)}
\vspace{0.1in}

DAC restricts access to objects based on the identity of the subject, i.e. the user or the group of the user. The accessibility of an object is determined by the owner of the object. The same idea has been adopted by Linux in file management.

As introduced earlier in database chapters, in DBMS, use syntax that looks like the following to grand and revoke access of an object from a subject.
\begin{lstlisting}
GRANT <privilege> ON <table/view> TO <subject>
REVOKE <privilege> ON <table/view> FROM <subject>
\end{lstlisting}
In practice, it is common that the database manager sets up different set of views for different user groups, each set of views containing everything that the user group requires. Grant access to only the associated views to the user groups. This can hopefully prevent a user from accessing data confidential to him.

The problem of DAC is that it can ``lose control'' sometimes, making a user bypassing the restriction. For example, a user who has been revoked from access may still be able to access the data if he had created a procedure that reads the data, and his access to that procedure is not revoked. Many DBMS tries to provide some protection against this, for example, by integrating security labels into the SQL that the user injects. When a user execute an SQL command, in the backend the SQL command is ``reformulated'' to contain user authentication information. In a good implementation, this security mechanism should be made transparent to the user.

Another challenge is that sometimes the user needs some aggregated information that can only be derived if he has access to data confidential to him. For example, consider a table that stores the scores of a class. A student wants to check his score as well as the average score of the class. If the access of the student is limited to his score alone, he will not be able to get the average score of the class.

\vspace{0.1in}
\noindent \textbf{Mandatory Access Control (MAC)}
\vspace{0.1in}

Different from DAC where the owner of an object choose his preference of who can access the data and the preference can change case by case, in MAC the rules are enforced by the system administrator consistently and globally. The user cannot overwrite security policy even on his own data. This reduces the chance of human error, hence providing a higher level of security. The cost is the flexibility in the user experience, and the complexity of setting up global rules especially on a large database. MAC is often used in government database where huge amount of confidential and sensitive data is managed with heavy responsibility.

\subsection{Multi-level Security DBMS}

In multi-level security DBMS (MLS), also know as multilayer DBMS, each piece of data in a database is associated with a security level that reads like ``unclassified'', ``confidential'', ``secret'', ``top secret'', etc. This security label is an compulsory attribute to the data, just like the primary key (but of course they serve different purposes). Users also have security labels that determines the level of accessibility. When querying the same database, different users will get different results based on the security level of each piece of data. For example, a general may get the number of solders in total, while the public can only get ``NULL'' in the return.

MLS is often used as part of MAC which is introduced earlier in this section.

In addition to query, the security labels also affect how \verb|INSERT| and other database manipulation commands work. Obviously, the user needs to have a higher layer (or at least equal) security label than the data, in order to insert, edit or remove it. 

There might be an interesting case where a row with higher security layer already exists in the table, and a lower security layer who cannot detect that data wants to insert into the table a new row with the same primary key. In a normal database, this operation would have been rejected due to the duplication of primary key. However, in MLS, this action is granted. Otherwise, the user with lower security layer would sense the existence of the higher security layer data. This technology is known as polyinstantiation, a method used to avoid covert channels by allowing multiple rows with the same primary key but with different data, based on different security levels. Polyinstantiation occurs when multiple rows in table appear to have the same primary key when viewed at different security levels.

Covert channel refers to a ``disguised'' channel that transfer information between entities while violating the security policy. In many cases, the channel is built from a list of operations, all of which legitimate by itself alone. These operations, when combined together, creates this unexpected bug outbound designer's intension. An example of a convert channel is as follows.
\begin{enumerate}
  \item Entity A, with higher privileges, encodes secret information in binary format.
  \item Periodically, entity A change the permission of a file that can be sensed by entity B. The encoded binary format is used to setup the permissions.
  \item Entity B listens to the permission of that file to obtain the binary format data.
  \item Entity B decodes the binary format to obtain the secret information.
\end{enumerate}
By doing the above, entity A is able to transfer a secret information to entity B who has a lower layer security label and should not touch the data. Notice that entity A may not be a human traitor, but a malware program.

Covert channel uses unintended system mechanisms for communication, which is often low efficient. As a result, the bandwidth of the covert channel is often much lower than a regular communication channel. Besides, a fast covert channel is likely to be easier to detect, which is something that the hackers want to avoid.

\subsection{Security-Enhanced DBMS Solutions in a Glance}

To wrap up, different security-enhanced DBMS solutions are now available on the market as follows.

\vspace{0.1in}
\noindent \textbf{Normal DBMS on Security-Enhanced OS}
\vspace{0.1in}

In the early stage, no additional security features is added to the DBMS. It is just that the DBMS is running on a secure OS with MAC enabled. The database is often put into the group with highest sensitivity level. The problem of this implementation is that all users who legitimately access the database have to be in the same highest sensitivity group, which violates the principle of access control. The output of the database, whatever it might be, is considered generated from the highest sensitivity group, and needs to be audited each time before release. This severely adds human cost.

\vspace{0.1in}
\noindent \textbf{MLS}
\vspace{0.1in}

MLS, also known as ``trusted database'', adopts security-enhanced DBMS that uses security labels to mark all the data and users. It is secure and flexible, and the details have been introduced in the earlier section. The only obvious problem for MLS is that it is difficult to realize. For third-party MLS, the customer never know whether there is a backdoor, unless he check all the codes (millions of lines of codes) that realize the DBMS by himself, which is enormously tedious and sometimes impossible.

\vspace{0.1in}
\noindent \textbf{Security Wrapper}
\vspace{0.1in}

An alternative to MLS is use normal OS and DBMS, and add a ``filter'' between DBMS and the users. This filter serves as a wrapper to the DBMS, and it uses security stamps to manage data transmission. All the data stored in the database can be encoded, and only users with the correct keys can decode them. This forms an encoded database with security wrapper.

\vspace{0.1in}
\noindent \textbf{Distributed Database}
\vspace{0.1in}

Till this point, it can be seen that the key to database security is to prevent data leakage from high security tier databases to low security tier databases. MLS tries to label the data carefully to isolate high security tier data from low security tier data, while the secure OS and secure wrapper try to prevent low security tier user from accessing high security tier data. Distributed database is another approach trying to further isolate the data of different security tiers: use different database, or even run them on different machines, in the first place.

Some system runs multiple DBMS kernels concurrently on a single server, each kernel managing a security layer of data. The compromise of one kernel does not necessarily mean that other layers are compromised. This may make the DBMS safer, but it will create high computational burden to the system. High security sometimes means low efficiency, and low efficiency can be bad for commercialized databases. It is also a challenge how to balance the trade-off between security, efficiency and cost.

Other system runs multiple DBMS kernels on concurrently on multiple servers, each server in charge of a security tier. Low security tier databases are synchronized to high security tier servers (bot not wise versa). This architecture design is robust to covert channel, but the data consistency and availability becomes a challenge.

\subsection{Outsourced Database Security}

Some enterprise outsources the database management and maintenance to IT companies such as Microsoft, Oracle or other database service providers. Security challenges introduced by cloud/vendor-based database differ largely from on-premises databases mainly because the DBMS itself is not reliable.

An intuitive solution is of course to encrypt the data to be saved in the database, and this is indeed one of the popular ways implemented in outsourced database security. A downside of this is that when we want to retrieve data from the database using an SQL query, the DBMS may have a difficult time interpreting and filtering the data. ``Searchable encryption'' is required in this case. It allows filtering of data without decrypting it first. 

The result from the outsourced database needs to be audited, mainly to check the authenticity, completeness and freshness of the returned information. Part of the reason is that searchable encryption may fail to return the correct and complete result. More importantly, the underlying assumption is that we cannot entirely trust the DBMS in the first place.

Watermark can be used to identify the owner and authenticity of the data. It does not affect the normal usage of data, and it is hardly detectable by a third-party, but can be checked and audited conveniently by the party who assigned the watermark. Watermark shall be difficult to remove. The water mark is often added to the least significant bits of a numerical data.





