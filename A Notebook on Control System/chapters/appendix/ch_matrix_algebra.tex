\chapter{Matrix Algebra} \label{ch:matrixalgebra}

This appendix chapter reviews matrix algebra, one of the most fundamental and widely used mathematical tools in this notebook. Notice that this chapter only serves as a brief summary, and cannot replace the linear algebra textbook or notebook.

\section{Basic Operators}

Let $A$, $B$ be two square matrices of the same size. The following holds true
\begin{eqnarray}
|A| &=& |A^T| \nonumber \\
|AB| &=& |A||B| \nonumber
\end{eqnarray}
where $|\cdot|$ calculates the determinant of a square matrix. Furthermore, if both $A$, $B$ are nonsingular,
\begin{eqnarray}
	(AB)^{-1} = B^{-1}A^{-1} \nonumber
\end{eqnarray}

Let $A\in\mathbb{R}^{p\times n}$, $B\in\mathbb{R}^{n\times q}$. The following holds true
\begin{eqnarray}
(AB)^T &=& B^TA^T \nonumber
\end{eqnarray}
Furthermore, if $p=q$,
\begin{eqnarray}
	\textup{trace}(AB) &=& \textup{trace}(BA) \nonumber
\end{eqnarray}

The \textit{Kronecker product} of two matrix $A\in\mathbb{R}^{m\times n}$, $B\in\mathbb{R}^{p\times q}$ is defined as follows.
\begin{eqnarray}
	A\otimes B &=& [a_{ij}B] \in \mathbb{R}^{mp\times nq} \nonumber
\end{eqnarray}
where $a_{ij}$ is an element in $A$. Notice that some literature defines it as $[Ab_{ij}]$ instead, which is different from the earlier definition.

The stacking operator is defined as follows. Let $A\in\mathbb{R}^{m\times n}$,
\begin{eqnarray}
	s(A) = \left[\begin{array}{c}
		a_{1,1} \\
		\vdots \\
		a_{m, 1} \\
		\vdots \\
		a_{1, n} \\
		\vdots \\
		a_{m, n}
	\end{array}\right] \nonumber
\end{eqnarray}
i.e., it stacks the columns of $A$ into a column vector.

The following holds true
\begin{eqnarray}
	s(ABC) &=& s(C^T\otimes A)s(B) \nonumber
\end{eqnarray}
Let $A\in\mathbb{R}^{m\times n}$, $B\in\mathbb{R}^{p\times q}$,
\begin{eqnarray}
	|A\otimes B| &=& |A|^p|B|^m \nonumber
\end{eqnarray}

If $\lambda_i$ is an eigenvalue of $A$ with eigenvector $v_i$, then $\lambda_i^{-1}$ is an eigenvalue of $A^{-1}$ with the same eigenvector. This is because
\begin{eqnarray}
	Av_i &=& \lambda_iv_i \nonumber
\end{eqnarray}
implies that
\begin{eqnarray}
	\lambda^{-1}v_i &=& A^{-1}v_i \nonumber
\end{eqnarray}
For another matrix $B$ with eigenvalue $\mu_i$ with respect to eigenvector $w_i$, $\lambda_i\mu_i$ is an eigenvalue of $A\otimes B$ w.r.t eigenvector $v_i\otimes w_i$.

The well-known matrix inversion lemma is
\begin{eqnarray}
  (A + UCV)^{-1} &=& A^{-1} - A^{-1}U\left(C^{-1}+VA^{-1}U\right)^{-1}VA^{-1} \nonumber
\end{eqnarray}
whose proof can be obtained using blockwise matrix inversion, which is neglected here.

\section{Partitioned Matrix}

The following matrix $A$ is \textit{block diagonal} if
\begin{eqnarray}
	A &=& \left[\begin{array}{ccc}
		\ddots &  & 0 \\
		 & A_{ii} &  \\
		0 &  & \ddots
	\end{array}\right] \nonumber
\end{eqnarray}
In this case, $A$ can be denoted by $A=\textup{diag}(\ldots, A_{ii}, \ldots)$. Furthermore, if $A_{ii}$ is square matrix, $A$ becomes square matrix, and
\begin{eqnarray}
	|A| &=& \Pi_{i}|A_{ii}| \nonumber
\end{eqnarray}
If $A$ is nonsingular,
\begin{eqnarray}
	A^{-1} &=& \textup{diag}(\ldots, A_{ii}^{-1}, \ldots) \nonumber
\end{eqnarray}

\section{Quadratic Forms}

Let $x\in\mathbb{R}^n$ a (column) vector. The square of the Euclidean norm is
\begin{eqnarray}
	||x||^2 &=& x^Tx \nonumber
\end{eqnarray}
Let $S$ be a nonsingular transformation, and $P=S^TS$ a symmetric matrix,
\begin{eqnarray}
	||x||_P^2 &=& ||Sx||^2 \nonumber \\
	&=& x^TS^TSx \nonumber \\
	&=& xPx^T \nonumber
\end{eqnarray}

The quadratic form of $x$ looks like the following
\begin{eqnarray}
	x^TQx \label{eq:quadraticform}
\end{eqnarray}
where $Q$ is real. Notice that $Q$ can be divided into symmetric and asymmetric parts
\begin{eqnarray}
	Q &=& Q_s + Q_a \nonumber
\end{eqnarray}
where
\begin{eqnarray}
	Q_s &=& (Q + Q^T) / 2 \nonumber \\
	Q_a &=& (Q - Q^T) / 2 \label{eq:asymmetricq}
\end{eqnarray}
and \eqref{eq:quadraticform} becomes
\begin{eqnarray}
	x^TQx &=& x^T(Q_s + Q_a)x \nonumber \\
	&=& x^TQ_sx \nonumber
\end{eqnarray}
where notice that $x^TQ_ax = 0$. This is because $x^TQ_ax=x^TQ_a^Tx=-x^TQ_ax$ since $x^TQ_ax$ is a scalar and from \eqref{eq:asymmetricq} $Q_a^T=-Q_a$.

In \eqref{eq:quadraticform}, we say $Q>0$ is positive definite if for all nonzero $x$, $x^TQx>0$. This is equivalent of saying that for all eigenvalues $\lambda_i$ of $Q$, $\lambda_i>0$. Semi-positive definite, semi-negative definite, and negative definite can be defined similarly.

\section{Matrix Calculus}

Matrix calculus defines how the differential of a vector/matrix w.r.t another vector/matrix is defined.

Consider $s$ being a scalar and $x\in\mathbb{R}^n$ a factor. Let $x$ be a vector function of $s$. The differential of $x$ w.r.t. $s$ is defined as follows.
\begin{eqnarray}
  \dfrac{dx}{ds} &=& \left[\begin{array}{c}
                             \dfrac{dx_1}{ds} \\
                             \vdots \\
                             \dfrac{dx_n}{ds}
                           \end{array}\right] \nonumber
\end{eqnarray}
If $s$ is a function of $x$, the gradient of $s$ w.r.t. $x$ has two forms, the gradient form where the result is a column vector, and the Jacobian form where the result is a row vector. They are transpose of each other. For example, the gradient form is
\begin{eqnarray}
   \dfrac{\partial s}{\partial x} = \nabla s
  = \left[\begin{array}{c}
           \dfrac{\partial s}{\partial x_1} \\
           \vdots \\
           \dfrac{\partial s}{\partial x_n}
           \end{array}\right] \label{eq:vectordiff}
\end{eqnarray}
and
\begin{eqnarray}
ds &=& \left(\nabla s\right)^Tdx \nonumber
\end{eqnarray}
where $dx = [dx_1, \ldots, dx_n]^T$. The Jacobian form denoted by $J(s)$ is defined as the transpose of \eqref{eq:vectordiff}. Furthermore, if $s$ is a function of multiple vectors, for example both $x$ and $y$,
\begin{eqnarray}
ds &=& \left(\dfrac{\partial s}{\partial x}\right)^Tdx + \left(\dfrac{\partial s}{\partial y}\right)^Tdy \nonumber
\end{eqnarray}
The \textit{Hessian} of a scalar $s$ w.r.t. a vector $x$ is the second derivative
\begin{eqnarray}
  \dfrac{\partial^2s}{\partial x^2} &=& \left[\dfrac{\partial^2s}{\partial x_i \partial x_j}\right] \nonumber
\end{eqnarray}

Let $f\in\mathbb{R}^{m}$ be a vector function of vector $x\in\mathbb{R}^{n}$. The jacobian of $f$ w.r.t. $x$ is a $m\times n$ matrix given by
\begin{eqnarray}
  \dfrac{\partial f}{\partial x} &=& \left[\begin{array}{ccc}
                                             \dfrac{\partial f_1}{\partial x_1} & \ldots & \dfrac{\partial f_1}{\partial x_n} \\
                                             \vdots & \ddots & \vdots \\
                                             \dfrac{\partial f_m}{\partial x_1} & \ldots & \dfrac{\partial f_m}{\partial x_n}
                                           \end{array}\right] \nonumber
\end{eqnarray}

Let $A$ be a non-singular square matrix of function of time $t$.
\begin{eqnarray}
  \dfrac{d}{dt}\left(A^{-1}\right) &=& -A^{-1}\dfrac{d}{dt}\left(A\right)A^{-1} \nonumber
\end{eqnarray}
Notice that in references, the differential of a variable $x$ or a matrix $A$ w.r.t. time can also be denoted as $\dot{x}$ and $\dot{A}$, respectively.

Let $y$, $A$ be constant vector and matrix, respectively, with dimensions so that the following expressions make sense. Then we have
\begin{eqnarray}
  && \dfrac{\partial}{\partial x}\left(y^Tx\right) = \dfrac{\partial}{\partial x}\left(x^Ty\right) = y \nonumber \\
  && \dfrac{\partial}{\partial x}\left(Ax\right) = A \nonumber \\
  && \dfrac{\partial}{\partial x}\left(y^TAx\right) = \dfrac{\partial}{\partial x}\left(x^TA^Ty\right) = A^Ty \nonumber \\
  && \dfrac{\partial}{\partial x}\left(y^Tf(x)\right) = \dfrac{\partial}{\partial x}\left(f(x)^Ty\right) = \left(\dfrac{\partial f}{\partial x}\right)^Ty \nonumber \\
  && \dfrac{\partial}{\partial x}\left(x^TAx\right) = Ax + A^T x \nonumber \\
  && \dfrac{\partial^2}{\partial x^2}\left(x^TAx\right) = A + A^T \nonumber
\end{eqnarray}
In a special case where $Q$ is symmetric,
\begin{eqnarray}
  \dfrac{\partial}{\partial x}\left(x^TQx\right) &=& 2Qx \nonumber \\
  \dfrac{\partial^2}{\partial x^2}\left(x^TQx\right) &=& 2Q \nonumber \\
  \dfrac{\partial}{\partial x}\left((x-y)^TQ(x-y)\right) &=& 2Q(x-y) \nonumber \\
  \dfrac{\partial^2}{\partial x^2}\left((x-y)^TQ(x-y)\right) &=& 2Q \nonumber
\end{eqnarray}

Furthermore, if $f$, $g$ are two vector functions of vector $x$,
\begin{eqnarray}
  \dfrac{\partial}{\partial x}\left(f^Tg\right) &=& \left(\dfrac{\partial}{\partial x}f\right)^Tg + \left(\dfrac{\partial}{\partial x}g\right)^Tf \nonumber
\end{eqnarray}
