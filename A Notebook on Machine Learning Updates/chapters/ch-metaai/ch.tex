\chapter{Meta AI} \label{ch:metaai}

Meta, formerly known as Facebook, provides variety of solutions for AI, including computer vision, stream data processing, natural language processing, and many more.

\section{General Introduction}

\subsection{Background}

\subsection{Business Model}

\section{PyTorch}

\section{LLaMA}

LLaMA is the large language model published by Meta AI. The associated paper that introduced LLaMA for the first time was published in February 2023 on ArXiv as \cite{touvron2023llama}. The initial LLaMA is trained on publicly available
datasets exclusively. 

Different levels of complexity of the models are provided. Like many other LLMs, Meta AI provides charged service for fine-tuning the model. There is also a re-write of LLaMA, namely Lit-LLaMA, that is completely open source under Apache 2.0 license. Both LLaMA and Lit-LLAMA can be fine-tuned via tools such as LoRA and Stanford Alpaca.

Comparing with other models such as ChatGPT-3, LLaMA is claimed to use smaller models (with less number of parameters) to achieve the same performance.

\subsection{LLaMA: Open and Efficient Foundation Language Models}

