\chapter*{Foreword}
If software and e-books can be made completely open-source, why not a notebook?

This brings me back to the summer of 2009 when I started my third year as a high school student in Harbin No. 3 High School. In the end of August when the results of Gaokao (National College Entrance Examination of China, annually held in July) are released, people from photocopy shops would start selling notebooks photocopies that they claim to be from the top scorers of the exam. Much curious as I was about what these notebooks look like, never have I expected myself to actually learn anything from them, mainly for the following three reasons.

First of all, some (in fact many) of these notebooks were more difficult to understand than the textbooks. I guess we cannot blame the top scorers for being so smart that they sometimes make things extremely brief or overwhelmingly complicated.

Secondly, why would I want to adapt to notebooks of others when I had my own notebooks which in my opinion should be just as good as theirs.

And lastly, as a student in the top-tier high school myself, I knew that the top scorers are probably my schoolmates or even friends. Why would I pay money to a complete stranger in a photocopy shop for my friends' notebook, rather than requesting a copy from them directly?

However, my mind changed after becoming an undergraduate student in 2010. There were so many modules and materials to learn for a college student, and as an unfortunate result, students were often distracted from digging deeply into a module (For those who were still able to do so, you have my highest respect). The situation became worse when I started pursuing my Ph.D. in 2014. As I had to focus on specific research areas entirely, I could hardly split much time on other irrelevant but still important and interesting contents.

In order to make a difference, I started enforcing myself reading articles beyond my comfort zone, which ended up motivating me to take notes to consolidate the knowledge. I used to work with hand-written notebooks. My very first notebook was on \textit{Numerical Analysis}, an entrance level module for engineering background graduate students. Till today I still have dozens of these notebooks on my bookshelf. Eventually, it came to me: why not digitizing them, making them accessible online and open-source and letting everyone read and edit it?

\noindent ---

\noindent As most of the open-source software, this notebook does not come with any ``warranty'' of any kind, meaning that there is no guarantee that everything in this notebook is correct, and it is not peer reviewed. \textbf{Do NOT cite this notebook in your academic research paper or book!} If you find anything helpful here with your research, please trace back to the origin of the knowledge and confirm by yourself.

This notebook is suitable as:
\begin{itemize}
  \item a quick reference guide;
  \item a brief introduction for beginners of an area;
  \item a ``cheat sheet'' for students to prepare for the exam or for lecturers to prepare the teaching materials.
\end{itemize}

This notebook is NOT suitable as:
\begin{itemize}
  \item a direct research reference;
  \item a replacement of the textbook;
\end{itemize}
because as explained the notebook is NOT peer reviewed and after all, it is more of a notebook than a book. It is meant to be easy to read, not to be comprehensive.

\vspace{0.1in}
\noindent \rule{1in}{0.4pt}
\vspace{0.1in}

Although this notebook is open-source, the reference materials of this notebook, including textbooks, journal papers, conference proceedings, etc., may not be open-source. Very likely many of these reference materials are licensed or copyrighted. Please legitimately access these materials and properly use them, should you decided to trace the origin of the knowledge.

Some of the figures in this notebook are plotted using Excalidraw, a very interesting tool to emulate hand drawings. The Excalidraw project can be found on GitHub, \textit{excalidraw/excalidraw}. Other figures may come from MATLAB, R, Python, and other computation engines. The source code to reproduce the results are intended to be included in the same repository of the notebook, but there might be exceptions.

\vspace{0.1in}
\noindent \rule{1in}{0.4pt}
\vspace{0.1in}

This work might have benefited from the assistance of large language models, which are used exclusively for editing purposes such as correcting grammar and rephrasing sentences, without introducing new content, generating novel information, or changing the original intent of the text.

