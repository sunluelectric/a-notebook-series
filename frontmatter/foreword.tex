\chapter*{Foreword}
If software and e-books can be made completely open-source, why not a notebook?

This brings me back to the summer of 2009 when I started my third year as a high school student in Harbin No. 3 High School. In the end of August when the results of Gaokao (National College Entrance Examination of China, annually held in July) are released, people from photocopy shops would start selling notebooks photocopies that they claim to be from the top scorers of the exam. Much curious as I was about what these notebooks look like, never have I expected myself to actually learn anything from them, mainly for the following three reasons.

First of all, some (in fact many) of these notebooks were more difficult to understand than the textbooks. I guess we cannot blame the top scorers for being so smart that they sometimes make things extremely brief or overwhelmingly complicated.

Secondly, why would I want to adapt to notebooks of others when I had my own notebooks which in my opinion should be just as good as theirs.

And lastly, as a student in the top-tier high school myself, I knew that the top scorers of the coming year would probably be a schoolmate or a classmate. Why would I want to pay that much money to a complete stranger in a photocopy shop for my friend's notebook, rather than requesting a copy from him or her directly?

However, my mind changed after becoming an undergraduate student in 2010. There were so many modules and materials to learn for a college student, and as an unfortunate result, students were often distracted from digging deeply into a module (For those who were still able to do so, you have my highest respect). The situation became worse when I started pursuing my Ph.D. in 2014. As I had to focus on specific research areas entirely, I could hardly split much time on other irrelevant but still important and interesting contents.

This motivated me to start reading and taking notebooks for selected books and articles, just to force myself to spent time learning new subjects out of my comfort zone. I used to take hand-written notebooks. My very first notebook was on \textit{Numerical Analysis}, an entrance level module for engineering background graduate students. Till today I still have on my hand dozens of these notebooks. Eventually, one day it suddenly came to me: why not digitalize them, and make them accessible online and open-source, and let everyone read and edit it?

\noindent ---

\noindent As most of the open-source software, this notebook (and it applies to the other notebooks in this series as well) does not come with any ``warranty'' of any kind, meaning that there is no guarantee for the statement and knowledge in this notebook to be absolutely correct as it is not peer reviewed. \textbf{Do NOT cite this notebook in your academic research paper or book!} Of course, if you find anything helpful with your research, please trace back to the origin of the citation and double confirm it yourself, then on top of that determine whether or not to use it in your research.

This notebook is suitable as:
\begin{itemize}
  \item a quick reference guide;
  \item a brief introduction for beginners of the module;
  \item a ``cheat sheet'' for students to prepare for the exam (Don't bring it to the exam unless it is allowed by your lecturer!) or for lecturers to prepare the teaching materials.
\end{itemize}

This notebook is NOT suitable as:
\begin{itemize}
  \item a direct research reference;
  \item a replacement to the textbook;
\end{itemize}
because as explained the notebook is NOT peer reviewed and it is meant to be simple and easy to read. It is not necessary brief, but all the tedious explanation and derivation, if any, shall be ``fold into appendix'' and a reader can easily skip those things without any interruption to the reading experience.

\noindent ---

Although this notebook is open-source, the reference materials of this notebook, including textbooks, journal papers, conference proceedings, etc., may not be open-source. Very likely many of these reference materials are licensed or copyrighted. Please legitimately access these materials and properly use them.

Some of the figures in this notebook is drawn using Excalidraw, a very interesting tool for machine to emulate hand-writing. The Excalidraw project can be found in GitHub, \textit{excalidraw/excalidraw}.

