\chapter{R (Part II: Advanced Data Analysis)} \label{ch:r2}

This chapter introduces advanced skills of using R language, including data preparation techniques, the use of list, etc.

\section{Data Preparation} \label{ch:r2:sec:datapreparation}

The data downloaded from sensors usually needs to go through pre-processing procedures such as filtering, normalization, etc., before it can be used by a controller, an AI engine, or for further statistics analysis. Data preparation including data tidy is one of the most tedious and time consuming parts when using R for data analysis. The section introduces useful techniques helpful with data preparation.

\subsection{Data Type Conversion}

It is important that the data types of all the columns meet expectation, especially for numeric and factor (categorical) data types. Use \verb|str(<df>)| to check the column data types of a data frame, and if necessary convert data types using the following commands.
\begin{lstlisting}
<df>$<column> <- factor(<df>$<column>) # character/numeric to factor
<df>$<column> <- as.numeric(<df>$<column>) # character to numeric
<df>$<column> <- as.numeric(as.character(<df>$<column>)) # factor to numeric
\end{lstlisting}

Notice that when converting factor type to other types, R may deal with the factor using the underlying ``factorization integers'' instead of the factor item names. An example is given below. It can be seen that the original \verb|5.1|, after being converted to factor then back to numeric, becomes \verb|9|. This is because the factorization integer for \verb|5.1| is \verb|9|, as shown by printing \verb|my_factor| to the console.
\begin{lstlisting}
> library(datasets)
> iris$Sepal.Length
  [1] 5.1 4.9 4.7 4.6 5.0 5.4 4.6 5.0 4.4 4.9 5.4 4.8 4.8 4.3 5.8 5.7
 [17] 5.4 5.1 5.7 5.1 5.4 5.1 4.6 5.1 4.8 5.0 5.0 5.2 5.2 4.7 4.8 5.4
 [33] 5.2 5.5 4.9 5.0 5.5 4.9 4.4 5.1 5.0 4.5 4.4 5.0 5.1 4.8 5.1 4.6
 [49] 5.3 5.0 7.0 6.4 6.9 5.5 6.5 5.7 6.3 4.9 6.6 5.2 5.0 5.9 6.0 6.1
 [65] 5.6 6.7 5.6 5.8 6.2 5.6 5.9 6.1 6.3 6.1 6.4 6.6 6.8 6.7 6.0 5.7
 [81] 5.5 5.5 5.8 6.0 5.4 6.0 6.7 6.3 5.6 5.5 5.5 6.1 5.8 5.0 5.6 5.7
 [97] 5.7 6.2 5.1 5.7 6.3 5.8 7.1 6.3 6.5 7.6 4.9 7.3 6.7 7.2 6.5 6.4
[113] 6.8 5.7 5.8 6.4 6.5 7.7 7.7 6.0 6.9 5.6 7.7 6.3 6.7 7.2 6.2 6.1
[129] 6.4 7.2 7.4 7.9 6.4 6.3 6.1 7.7 6.3 6.4 6.0 6.9 6.7 6.9 5.8 6.8
[145] 6.7 6.7 6.3 6.5 6.2 5.9
> my_factor <- factor(iris$Sepal.Length)
> my_numeric <- as.numeric(my_factor)
> my_numeric
  [1]  9  7  5  4  8 12  4  8  2  7 12  6  6  1 16 15 12  9 15  9 12  9
 [23]  4  9  6  8  8 10 10  5  6 12 10 13  7  8 13  7  2  9  8  3  2  8
 [45]  9  6  9  4 11  8 28 22 27 13 23 15 21  7 24 10  8 17 18 19 14 25
 [67] 14 16 20 14 17 19 21 19 22 24 26 25 18 15 13 13 16 18 12 18 25 21
 [89] 14 13 13 19 16  8 14 15 15 20  9 15 21 16 29 21 23 33  7 31 25 30
[111] 23 22 26 15 16 22 23 34 34 18 27 14 34 21 25 30 20 19 22 30 32 35
[133] 22 21 19 34 21 22 18 27 25 27 16 26 25 25 21 23 20 17
> typeof(my_factor)
[1] "integer"
> my_factor
  [1] 5.1 4.9 4.7 4.6 5   5.4 4.6 5   4.4 4.9 5.4 4.8 4.8 4.3 5.8 5.7
 [17] 5.4 5.1 5.7 5.1 5.4 5.1 4.6 5.1 4.8 5   5   5.2 5.2 4.7 4.8 5.4
 [33] 5.2 5.5 4.9 5   5.5 4.9 4.4 5.1 5   4.5 4.4 5   5.1 4.8 5.1 4.6
 [49] 5.3 5   7   6.4 6.9 5.5 6.5 5.7 6.3 4.9 6.6 5.2 5   5.9 6   6.1
 [65] 5.6 6.7 5.6 5.8 6.2 5.6 5.9 6.1 6.3 6.1 6.4 6.6 6.8 6.7 6   5.7
 [81] 5.5 5.5 5.8 6   5.4 6   6.7 6.3 5.6 5.5 5.5 6.1 5.8 5   5.6 5.7
 [97] 5.7 6.2 5.1 5.7 6.3 5.8 7.1 6.3 6.5 7.6 4.9 7.3 6.7 7.2 6.5 6.4
[113] 6.8 5.7 5.8 6.4 6.5 7.7 7.7 6   6.9 5.6 7.7 6.3 6.7 7.2 6.2 6.1
[129] 6.4 7.2 7.4 7.9 6.4 6.3 6.1 7.7 6.3 6.4 6   6.9 6.7 6.9 5.8 6.8
[145] 6.7 6.7 6.3 6.5 6.2 5.9
35 Levels: 4.3 4.4 4.5 4.6 4.7 4.8 4.9 5 5.1 5.2 5.3 5.4 5.5 ... 7.9
\end{lstlisting}

When converting factor to other types, special caution is required. To convert a factor to other types such as numeric, consider converting it to character first as given in the following example.
\begin{lstlisting}
> my_numeric <- as.numeric(as.character(my_factor))
> my_numeric
  [1] 5.1 4.9 4.7 4.6 5.0 5.4 4.6 5.0 4.4 4.9 5.4 4.8 4.8 4.3 5.8 5.7
 [17] 5.4 5.1 5.7 5.1 5.4 5.1 4.6 5.1 4.8 5.0 5.0 5.2 5.2 4.7 4.8 5.4
 [33] 5.2 5.5 4.9 5.0 5.5 4.9 4.4 5.1 5.0 4.5 4.4 5.0 5.1 4.8 5.1 4.6
 [49] 5.3 5.0 7.0 6.4 6.9 5.5 6.5 5.7 6.3 4.9 6.6 5.2 5.0 5.9 6.0 6.1
 [65] 5.6 6.7 5.6 5.8 6.2 5.6 5.9 6.1 6.3 6.1 6.4 6.6 6.8 6.7 6.0 5.7
 [81] 5.5 5.5 5.8 6.0 5.4 6.0 6.7 6.3 5.6 5.5 5.5 6.1 5.8 5.0 5.6 5.7
 [97] 5.7 6.2 5.1 5.7 6.3 5.8 7.1 6.3 6.5 7.6 4.9 7.3 6.7 7.2 6.5 6.4
[113] 6.8 5.7 5.8 6.4 6.5 7.7 7.7 6.0 6.9 5.6 7.7 6.3 6.7 7.2 6.2 6.1
[129] 6.4 7.2 7.4 7.9 6.4 6.3 6.1 7.7 6.3 6.4 6.0 6.9 6.7 6.9 5.8 6.8
[145] 6.7 6.7 6.3 6.5 6.2 5.9
\end{lstlisting}
where \verb|my_factor| is generated previously.

It is possible for some columns in the data frame to look like a factor and a character string, but indeed should be handled as numeric values. For example, \verb|S$6,125.50| in many occasions should be treated just as \verb|6125.5|. These factor or character values cannot be converted to numeric values directly.

In such cases, consider using \verb|sub()| or \verb|gsub()| to replace patterns in a character, then convert it into numeric values. Notice that \verb|sub()| replaces only the first encounter of the pattern, while \verb|gsub()| replaces all the encounters. An example of using \verb|gsub()| is given below.
\begin{lstlisting}
> money_character <- c("S$6,273.15", "S$215.3", "S$8,987,756.00")
> typeof(money)
[1] "character"
> a <- gsub(",", "", money) # replace "," with ""
> a <- gsub("S\\$", "", a) # replace "S$" with ""
> money_numeric <- as.numeric(a)
> money_numeric
[1]    6273.15     215.30 8987756.00
> typeof(money_numeric)
[1] "double"
\end{lstlisting}
where notice that \verb|$| is a special character defined in R, and to escape from that \verb|\\$| is used. Notice that applying \verb|sub()| and \verb|gsub()| on a factor automatically converts it to character as a hidden step.

\subsection{Handling Missing Data}

There can be missing data in the data frame. There are a few ways to deal with missing data as follows.
\begin{itemize}
	\item If the missing data can be derived from other columns, derive the missing data and fill in the blanks.
	\item If the missing data does not affect the rest analysis, leave it blank.
	\item Delete the row.
	\item Use interpolations to fill in the blank.
	\item Use correlations and similarities to fill in the blank.
	\item Argument a new column add a ``data-missing'' flag to that row.
\end{itemize}

\vspace{0.1in}
\noindent \textbf{Flag Missing Data using \texttt{NA}}
\vspace{0.1in}

In R, \verb|NA| is a special variable used to indicate a missing value. A general idea is to ``flag'' the missing data in the original data source, what format it may look like, using \verb|NA| during or after the data importing. After that, use a special program in R to filter \verb|NA| and deal with them separately. Sometimes a blank string \verb|""| that we would expected to be treated as \verb|NA| is not treated as so. To fix that, while importing the data frame (say, from a CSV file), use the following
\begin{lstlisting}
df <- read.csv("<csv-name>", na.string=c("<pattern>", ...))
\end{lstlisting}
where \verb|"<pattern>"| are the patterns in the original file to be replaced by \verb|NA|, for instance, \verb|""|, \verb|"ERROR"|, etc.

\vspace{0.1in}
\noindent \textbf{Locate and Filter \texttt{NA}}
\vspace{0.1in}

Notice that \verb|NA| is treated as a logical data type in addition to \verb|TURE| and \verb|FALSE| in a logical expression. These operations involving \verb|NA| often return \verb|NA|. Examples are given below.
\begin{lstlisting}
> typeof(NA)
[1] "logical"
> TRUE == 1 # TRUE is equivalent with 1
[1] TRUE
> TRUE == 2
[1] FALSE
> FALSE == 0 # FALSE is equivalent with 0
[1] TRUE
> FALSE == -1
[1] FALSE
> TRUE == FALSE
[1] FALSE
> NA == NA
[1] NA
> NA == TRUE
[1] NA
> NA == FALSE
[1] NA
\end{lstlisting}
This applies to filtering. In filtering, when a variable of value \verb|NA| is asserted with a criterion, the return is most likely \verb|NA|. The filter often does not know how to deal with \verb|NA|, hence it would simply return the rows with \verb|NA| anyway. This can become inconvenient sometimes. An example is given below.
\begin{lstlisting}
# create data frame
region <- rep(c("CBD","City","Suburbs"), each = 5)
vec_size <-  list(vec_size_cbd = rnorm(5, 75, 10),
                 vec_size_city = rnorm(5, 100, 15),
                 vec_size_suburbs = rnorm(5, 150, 25))
vec_price <- list(vec_price_cbd = vec_size$vec_size_cbd*rnorm(5, 12500, 2500),
                 vec_price_city = vec_size$vec_size_city*rnorm(5, 7500, 1000),
                 vec_price_suburbs = vec_size$vec_size_suburbs*rnorm(5, 5000, 1000))
vec_size <- unlist(vec_size)
vec_size[sample(1:15, 2)] = NA
vec_price <- unlist(vec_price)
vec_price[sample(1:15, 2)] = NA

mortgage_price <- data.frame(Region = region,
                             Size = vec_size,
                             Price = vec_price)
mortgage_price$Region <- as.factor(mortgage_price$Region)
mortgage_price$Price.Unit <- mortgage_price$Price / mortgage_price$Size
print(mortgage_price)
# find price > 750000
mortgage_price[mortgage_price$Price>750000,]
\end{lstlisting}
The return is as follows.
\begin{lstlisting}
                   Region      Size     Price Price.Unit
vec_size_cbd1         CBD  73.47779  947130.9  12890.031
vec_size_cbd2         CBD        NA  678842.3         NA
vec_size_cbd3         CBD  85.06748 1261029.7  14823.875
vec_size_cbd4         CBD  92.35454        NA         NA
vec_size_cbd5         CBD  71.06649 1276168.6  17957.388
vec_size_city1       City  68.59874  491912.0   7170.861
vec_size_city2       City 118.39441  804794.5   6797.572
vec_size_city3       City        NA  534591.5         NA
vec_size_city4       City  95.57428  583044.7   6100.436
vec_size_city5       City  74.45356  468788.0   6296.382
vec_size_suburbs1 Suburbs 136.88432  939436.6   6862.997
vec_size_suburbs2 Suburbs 136.57799  810070.0   5931.190
vec_size_suburbs3 Suburbs 189.66586  561089.2   2958.303
vec_size_suburbs4 Suburbs 195.97476  913572.2   4661.683
vec_size_suburbs5 Suburbs 190.63082        NA         NA
\end{lstlisting}
\begin{lstlisting}
                   Region      Size     Price Price.Unit
vec_size_cbd1         CBD  73.47779  947130.9  12890.031
vec_size_cbd3         CBD  85.06748 1261029.7  14823.875
NA                   <NA>        NA        NA         NA
vec_size_cbd5         CBD  71.06649 1276168.6  17957.388
vec_size_city2       City 118.39441  804794.5   6797.572
vec_size_suburbs1 Suburbs 136.88432  939436.6   6862.997
vec_size_suburbs2 Suburbs 136.57799  810070.0   5931.190
vec_size_suburbs4 Suburbs 195.97476  913572.2   4661.683
NA.1                 <NA>        NA        NA         NA
\end{lstlisting}
In the above example, the intension of the program is to find all the houses with price larger than $750000$. The program is able to filter out those houses cheaper than the threshold. However, there are two rows of \verb|NA| returned, as explained earlier.







Use the following to filter for all rows with/without at least one \verb|NA|.
\begin{lstlisting}
<df>[complete.cases(<df>),] # all complete rows
<df>[!complete.cases(<df>),] # all incomplete rows
\end{lstlisting}
where \verb|complete.cases(<df>)| returns a list made up of \verb|TRUE| and \verb|FALSE| indicating whether the associated row is complete or now.





\section{Connectivity with Data Sources} \label{ch:r2:sec:datasource}

This section introduces the connectivity of R to the data sources, such as a file, or a database.







