\chapter*{Preface}

This notebook introduces probability, statistics, data science and engineering. They are the ``must-have'' ability in most, if not all, academic and industrial projects.

In Part I of the notebook, probability theory is introduced. Probability theory studies how likely an event is to occur. It offers rich models and tools to describe random values and stochastic events.

In Part II of the notebook, statistics is introduced. Statistics is a collection of methods to analyze and observe insights from data, verify statistics hypothesis and draw conclusions and predictions.

In Part III of the notebook, commonly used software and toolkits for statistics analysis and data science are introduced. Different from Parts I and II which focus more on theory, Part III focuses more on the tools to solve practical problems. 

Artificial Intelligence (AI) has become notably popular in recent years for data analysis. There is a separate notebook on introducing AI, and hence in this notebook AI-based data analysis is not introduced in details, but only briefly covered.

As a bonus, in Part IV, semantic web, the database framework defined under Web 3.0, is introduced. Semantic web does not necessarily contribute to solving a specific data science problem. It is rather a concept that allows convenient and flexible information storage and exchange, thus serving as a probable backbone to improve efficiency in data analysis.

Key references of this notebook are listed. Notice that these materials are so very widely cited here and there in the entire notebook that it becomes improbable to address them each time they are used.

For probability and statistics:
\begin{itemize}
  \item Spiegel, Murray, John Schiller, and Alu Srinivasan. \textit{Probability and statistics.} 2020.
  \item Dekking, Frederik Michel, et al., \textit{A Modern Introduction to Probability and Statistics: Understanding why and how.} Vol. 488. London: Springer, 2005.
\end{itemize}

For data science:
\begin{itemize}
  \item Kirill Eremenko, \textit{R Programming A-Z: R For Data Science With Real Exercises}, Udemy Course.
  \item Lakshmanan, Valliappa, Martin Görner, and Ryan Gillard. Practical Machine Learning for Computer Vision. " O'Reilly Media, Inc.", 2021.
  \item Jose Portilla, \textit{Complete TensorFlow 2 and Keras Deep Learning Bootcamp}, Udemy
\end{itemize}

Online materials such as tutorials from YouTube, Bilibili, etc., are also used in forming this notebook. Conversations with ChatGPT are included in Part IV where we discussed the advantages and shortages of semantic web. They are interesting and to some extent inspiring. 