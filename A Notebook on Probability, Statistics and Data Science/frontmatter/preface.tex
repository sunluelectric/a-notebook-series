\chapter*{Preface}

This notebook introduces probability and statistics, which are one of the most important, fundamental and commonly used skills in almost all science and engineering subjects.

In Part I of the notebook, probability theory is introduced. Probability theory studies how likely an event is to occur. It offers rich models and tools to describe random values and stochastic events.

In Part II of the notebook, statistics is introduced. Statistics is a collection of methods to analyze and observe insights from data, verify statistical hypotheses, and draw conclusions and predictions.

In Part III of the notebook, commonly used software and toolkit for statistical analysis and data science are introduced. Different from Parts I and II which focus more on theory, Part III focuses more on the tools to solve practical problems. 

Artificial intelligence has become notably popular in recent years for data analysis. There is a separate notebook introducing AI, and hence in this notebook AI-based data analysis is not introduced in detail, but only briefly covered.

As a bonus, in Part IV of the notebook the semantic web is introduced. Semantic web is both a philosophical concept and a data framework defined under Web 3.0. It does not necessarily contribute to solving a specific data science problem, but rather a concept that allows convenient and flexible information storage and exchange, thus serving as a probable backbone to improve efficiency in data analysis.

Key references of this notebook are listed below.

Probability and statistics:
\begin{itemize}
  \item Spiegel, Murray, John Schiller, and Alu Srinivasan. \textit{Probability and statistics.} 2020.
  \item Dekking, Frederik Michel, et al., \textit{A Modern Introduction to Probability and Statistics: Understanding why and how.} Vol. 488. London: Springer, 2005.
\end{itemize}

Data science:
\begin{itemize}
  \item Kirill Eremenko, \textit{R Programming A-Z: R For Data Science With Real Exercises}, Udemy Course.
  \item Lakshmanan, Valliappa, Martin Görner, and Ryan Gillard. \textit{Practical Machine Learning for Computer Vision}. O'Reilly Media, Inc., 2021.
  \item Jose Portilla, \textit{Complete TensorFlow 2 and Keras Deep Learning Bootcamp}, Udemy
\end{itemize}

