\chapter{Introduction to Artificial Intelligence}

This chapter discusses the concept and scope of artificial intelligence and gives a brief review of its development trend.

\section{Human-Level Artificial Intelligence}

\mync{Artificial intelligence}[AI] refers to the trail where we want to understand and build intelligent entities. In this section, the scope of AI is discussed. The different approaches and the development trend are briefly introduced.

\subsection{Scope}

It is sometimes not easy to give a universally consistent definition to AI because our understanding of intelligence is evolving over time. Intuitively, we look at ourselves and try to find out what makes human intelligent. Many concepts, models and even research areas have been proposed to explain how human intelligence is composed. For example, human has \mync{knowledge} which is a mechanism that allows us to remember facts and experience, and human can obtain new knowledge by either \mync{reasoning} which derives new knowledge from existing knowledge, or \mync{learning} which abstracts knowledge from examples or datasets. Human is \mync{rational} when making decisions, which means we do the ``right thing'' that is most beneficial to individuals or teams. Everything above, and probably many other factors that we are not aware of yet, jointly make human intelligent.

We want to build a system that can mimic human behavior and carry out human jobs. Ideally, the system should have human-level intelligence or even beyond. This level of intelligence is often known as \mync{artificial general intelligence}[AGI]. 

\mync{Turing test}, proposed by Alan Turing, is a widely accepted operational definition of (a portion of) AGI. A computer-based system passes the test if a human interrogator, after posing some questions, cannot tell whether the response come from a computer or a human. Obviously, to pass the Turing test, the computer should be able to process natural language, and should have knowledge representation, can do human-level reasoning, can make rational decision, and can learn from the conversation. From industrial perspective, a useful AGI system should be able to process not just writing language, but also audio and video, and to manipulate objects and move about.

As of this writing, we have not yet achieved AGI, although recent advent in large language models (see Part IV of this notebook for more details) has for a short time made us believe that we might be close to it.

It is worth mentioning that although there is not yet a comprehensive AI that is comparable with a human at every aspect, we do have AIs that are very good at specific tasks, such as classifications, processing videos and audios, playing board games, etc. They are not AGI, yet still useful and productive in practice. More of them will be introduced in the remaining of the notebook.

\subsection{Approaches}

There are several promising approaches to build AI and hopefully some of them will eventually lead us to AGI. Joint effort of different research subjects and backgrounds is required along the journey.

\subsection{Development Trend}

\section{AI Agent}

\subsection{Expectation}

\subsection{Agent Structure}