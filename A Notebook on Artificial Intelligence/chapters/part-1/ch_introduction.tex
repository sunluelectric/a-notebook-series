\chapter{Introduction to Artificial Intelligence}

This chapter discusses the concept and scope of artificial intelligence and gives a brief review of its development trend. Machine learning is a broader concept than artificial intelligence. Before the introduction of artificial intelligence, machine learning is briefly reviewed.

\section{Machine Learning}

\mync{Machine learning} refers to the field of study that gives computers the ability to learn without being explicitly programmed to do so. It learns from the data or experience it sees, and hence it is more capable of handling complicated stuff and more robust to exceptions. 

Depending on the application requirement and training method, there are at least the following $3$ types of machine learning problems.
\begin{itemize}
  \item \mync{Supervised learning}
  
  A sample is given in the form of $(x,y)$, where $x$ is the input features (usually in vector form) and $y$ the label or reference. The purpose of supervised learning is to build a machine learning system that estimates $y$ when receiving input $x$.
  
  Example: on a house price guide website, given the size and location of a house, estimate its price on the market.
  
  \item \mync{Reinforcement learning}
  
  A sample is given in the form of $(x, a, r)$ where $x$ is the state of the system, $a$ (optional) an action that the system takes, and $r$ the reward. The purpose of reinforcement learning is to build a machine learning system that determines the optimal action to take that maximizes the utility of the system. The utility often includes not only the immediate reward, but also future foreseeable benefits. Notice that different from supervised learning, the optimal action is not included directly in the sample during training. The machine needs to find out what the optimal action is by exploring the action space.
  
  Example: in an autonomous driving vehicle, given the current status of the road environment and the vehicle, decide what actions to take. 
  
  \item \mync{Unsupervised learning}
  
  A sample is given in the form of $x$ only the input corpus without any label, reference or rewards. The machine is expected to find the patterns hidden in $x$. The purpose of unsupervised learning is to build a machine learning system that categories a new input based on the detected patterns. 
  
  Example: in a auto-fill text completion tool, given the first part of a sentence, fill in the remainder of the sentence.
  
\end{itemize}


\section{Artificial Intelligence}

\mync{Artificial intelligence}[AI] refers to the intelligent entities we human have built to mimic ourselves, or part of ourselves. It is one of the many things that is achieved (or will be achieved) with machine learning.

The different approaches and the development trend are briefly introduced.

\subsection{Scope}

It is not easy to give a universally consistent definition to AI because our understanding of intelligence is evolving over time. 

We look at ourselves and try to find out what makes human intelligent. Many concepts, models and even research areas have been proposed to explain what human intelligence is composed of. For example, human has \mync{knowledge} which is a mechanism that allows us to remember facts and experience, and human can obtain new knowledge by either \mync{reasoning} which derives new knowledge from existing knowledge, or \mync{learning} which gets knowledge from examples, datasets or experiences. Human is usually \mync{rational} when making decisions, which means we do the ``right thing'' that is most beneficial to individuals or teams. Everything above, and probably many other factors that we are not aware of yet, jointly make human intelligent.

We want to build a system that can mimic human behavior and complete human jobs with the same or better quality. Ideally, the system should have human-level intelligence or even beyond. This level of intelligence is often known as \mync{artificial general intelligence}[AGI]. 

\mync{Turing test}, proposed by Alan Turing, is a widely accepted operational definition of (a portion of) AGI. A computer-based system passes the test if a human interrogator, after posing some questions, cannot tell whether the responses come from a computer or a human. Obviously, to pass the Turing test, the computer should be able to process natural languages, have knowledge representations, can do human-level reasoning, can make rational decisions, can learn from conversations, and can be generative and creative. A practically more useful AGI system should be able to process not just writing languages, but also audio and video.

As of this writing, we have not yet achieved AGI, although recent advent in \myabb{large language model}{LLM} (see Part IV of this notebook for more details) has for a short time made us believe that we might be close to it.

It is worth mentioning that although there is not yet a comprehensive AI that is comparable with a human at every aspect, we do have AIs that are very good at specific tasks, such as classifications, processing videos and audios, playing board games, autonomous driving, etc. They are not AGI, yet still useful and productive in practice, and have started playing important roles in human industry.

\subsection{Approaches}

There are several promising approaches to build AI and hopefully some of them will eventually lead us to AGI. Joint effort of different research subjects and backgrounds is required along the journey.

\subsection{Development Trend}

\section{AI Agent}

\subsection{Expectation}

\subsection{Agent Structure} 