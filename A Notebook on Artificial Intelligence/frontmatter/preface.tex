\chapter*{Preface}

Artificial Intelligence (AI) was included as part of the control system notebook, as in early ages it was mostly used as a system identification tool in control systems.

With the advent of graphical processing units (GPU) in the $1990$th and the Industry 4.0 initiatives in $2000$th, deep learning network with massive training data became possible, which significantly boosted the performance of artificial neural network (ANN)-based AI systems. Nowadays, AI has been growing rapidly with successful demonstrations of use cases such as computer vision and natural language processing.

Seeing that trend, AI relevant contents have been separated from the control system notebook and they are collected here in this notebook. 

\vspace{0.1in}
\noindent \rule{1in}{0.4pt}
\vspace{0.1in}

Special thanks go to the following materials, all of which have been very useful when drafting this notebook. Notice that the referenced contents from the following materials will not be listed separately in the Reference section in the end of the notebook.
\begin{itemize}
  \item Russell, Stuart J., and Peter Norvig. Artificial Intelligence: a Modern Approach. Pearson, 2016.
  \item Lakshmanan, Valliappa, Martin G\"orner, and Ryan Gillard. Practical Machine Learning for Computer Vision. ``O'Reilly Media, Inc.'', 2021. 
  \item Ed Donner, Ligency Team, LLM Engineering: Master AI, Large Language Models \& Agents. Udemy, 2025
  \item Ed Donner, Ligency Team, The Complete Agentic AI Engineering Course. Udemy, 2025
\end{itemize} 